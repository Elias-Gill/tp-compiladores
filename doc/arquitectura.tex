\section{Arquitectura}

El sistema cuenta con los siguientes módulos:

\begin{itemize}
	\item Tabla de sentimientos
	\item Tokenizador
	\item Analizador de sentimientos
	\item Reportes (menú principal)
\end{itemize}

Cada modulo esta diseñado para realizar funciones especificas de manera modular, por lo que
se pueden realizar distintas implementaciones siempre y cuando se sigan las interfaces comunes.

\subsection{Flujo del programa}
El programa inicia cargando la tabla con las palabras necesarias para realizar el análisis de
sentimientos. A continuación, se selecciona e instancia el tipo de tokenizador, según los
argumentos proporcionados al momento de ejecutar el programa.

Si se indica un archivo como argumento, su contenido es procesado automáticamente. En caso
contrario, el programa entra en modo interactivo, permitiendo al usuario ingresar texto
directamente por consola.

La entrada es primero tokenizada utilizando el tokenizador seleccionado. Con la lista completa
de tokens disponibles, aquellos que son clasificados como \texttt{desconocidos} son presentados
al usuario para su corrección o descarte. Para cada token desconocido, el usuario puede elegir
entre las siguientes opciones:

\begin{itemize}
	\item Ignorar el token.
	\item Añadirlo a la tabla de sentimientos.
	\item Corregirlo utilizando una de las palabras sugeridas (si existen).
\end{itemize}

En caso de optar por la corrección mediante sugerencias, se muestra una lista de palabras
similares, calculadas utilizando la distancia de Levenshtein.

Finalmente, se realiza el análisis de sentimiento utilizando los tokens generados. El resultado
del análisis se muestra por consola. Se guardan en archivos de texto plano tanto un resumen del
procesamiento como la lista de tokens resultantes del proceso de tokenizacion.

\subsection{Tabla de Sentimientos}
La tabla de sentimientos es la estructura que contiene las palabras y frases categorizadas por
tipo junto con los puntajes asociados a cada una. Estos puntajes son utilizados posteriormente
para calcular la polaridad del texto analizado.

Además de su rol como base de conocimiento, esta tabla incluye las siguientes funcionalidades
de conveniencia que facilitan su uso y actualización dinámica durante el análisis:

\begin{itemize}
	\item \textbf{Búsqueda de palabras o frases:} permite consultar rápidamente si una palabra
	      o frase está presente en la tabla y recuperar su puntaje asociado.

	\item \textbf{Cálculo de correcciones recomendadas:} Calcula sugerencias de posibles
	      correcciones utilizando la distancia de Levenshtein, proponiendo palabras similares ya
	      registradas.

	\item \textbf{Incorporación de nuevas entradas:} el usuario puede añadir palabras con
	      su respectivo puntaje de sentimiento, ampliando así progresivamente la tabla a medida
	      que se utiliza el programa.
\end{itemize}

\subsection{Tokenizador}
El tokenizador es el componente encargado de segmentar el texto de entrada en unidades léxicas
denominadas \emph{tokens}. Es capaz de identificar y clasificar distintos tipos de tokens,
tales como palabras simples, frases compuestas, signos de puntuación, números o símbolos
especiales. Además, posee mecanismos para resolver ambigüedades en aquellos casos donde
múltiples tipos pudieran coincidir con un mismo patrón textual.

El tokenizador recibe una entrada en forma de texto plano y retorna una lista estructurada de
tokens.

\subsubsection{Tokens}
Cada token incluye la siguiente información:

\begin{itemize}
	\item \textbf{Texto del token:} la representación literal extraída del texto original.
	\item \textbf{Tipo:} categoría asignada al token (por ejemplo, palabra conocida, palabra desconocida, puntuación, etc.).
	\item \textbf{Puntaje:} valor numérico asociado al token, obtenido de la tabla de sentimientos si corresponde.
\end{itemize}

\subsection{Analisis de sentimiento}
Una vez obtenida la lista completa de tokens, se procede al análisis de estos para determinar
su validez y relevancia en el contexto del análisis de sentimientos.

Si se detectan tokens de tipo \emph{desconocido} (que no se encuentran registrados en la tabla
de sentimientos), el sistema solicita al usuario elegir si ignorarlos o proceder a su
corrección de manera manual. En caso de que el usuario decida corregir los tokens desconocidos,
este módulo se encarga de gestionar todo el proceso de corrección.

El análisis de sentimiento se realiza de forma diferenciada para cada hablante, lo que permite
evaluar en detalle el desempeño individual del cliente y del agente dentro de la conversación.

\subsection{Reportes}
El módulo de reportes se encarga de generar archivos con los resultados del proceso de
tokenización y del análisis de sentimiento. Además, presenta un resumen conciso directamente en
la terminal.

%%%%%%%%%%%%%%%%%%%%%%%%%%%%%%%%%%%%%%%%%%%%%%%%%
% Decisiones clave en la parte de implemetacion %
%%%%%%%%%%%%%%%%%%%%%%%%%%%%%%%%%%%%%%%%%%%%%%%%%
\subsection{Decisiones clave}

\subsubsection{Tipos de tokens}
Los tipos de tokens identificados por el tokenizador son:

\begin{enumerate}
	\item {\footnotesize\textbf{TOKEN\_SALUDO}}: frase o palabra que corresponde a un saludo.
	\item {\footnotesize\textbf{TOKEN\_DESPEDIDA}}: frase o palabra de despedida.
	\item {\footnotesize\textbf{TOKEN\_IDENTIFICACION}}: indica la identificación de alguna de las partes.
	\item {\footnotesize\textbf{TOKEN\_PROHIBIDA}}: palabra o frase grosera.
	\item {\footnotesize\textbf{TOKEN\_SENTIMIENTO}}: palabra individual con un sentimiento asociado.
	\item {\footnotesize\textbf{TOKEN\_DESCONOCIDO}}: palabra no registrada en la tabla de sentimientos.
	\item {\footnotesize\textbf{TOKEN\_AGENTE}}: indica cambio de hablante a un agente.
	\item {\footnotesize\textbf{TOKEN\_CLIENTE}}: indica cambio de hablante a un cliente.
	\item {\footnotesize\textbf{TOKEN\_SIGNO\_PUNTUACION}}: representa signos de puntuación.
\end{enumerate}

\subsubsection{Detección de frases}
Como parte del diseño del sistema, se asumió el desafío de identificar no solo palabras
aisladas, sino también frases cortas como \textit{"buenos días"}, con el fin de facilitar la
identificación de protocolos como saludos o identificaciones.

Esto implica que el sistema debe resolver ambigüedades, por ejemplo, entre \textit{"buenos"}
(sentimiento) y \textit{"buenos días"} (saludo).

Esto influye directamente en la implementación del mecanismo de tokenizacion, dado que el
tokenizador debe de priorizar correctamente las frases completas por sobre palabras sueltas.

\subsubsection{Mecanismo de tokenizacion}
Dado el tiempo disponible, se desarrollaron dos tipos de tokenizadores: uno basado en autómatas
finitos deterministas (AFD) y otro basado en tablas hash.

Ambos implementan una interfaz común con el método \textbf{\textit{tokenizar()}}, el cual
devuelve una lista con los tokens analizados.

Se consideró la posibilidad de utilizar un método \textbf{\textit{next\_token()}}, pero se optó
por una solución más simple que retorne todos los tokens de una vez.

Gracias a su diseño modular, los tokenizadores son fácilmente intercambiables, lo que permite
realizar comparaciones sobre facilidad de mantenimiento, extensibilidad, implementación y
rendimiento.

\subsubsection{API de la tabla de sentimientos}
La tabla de sentimientos almacena los distintos tipos de frases (saludos, sentimientos, etc.)
en diccionarios separados. Además, cuenta con funciones auxiliares, entre ellas la más
importante es \textbf{\textit{buscar\_palabra()}}, utilizada especialmente por el tokenizador
basado en hashmaps.

Esta función garantiza el siguiente orden de prioridad al buscar una palabra:

\begin{enumerate}
	\item \footnotesize{TOKEN\_SALUDO}
	\item \footnotesize{TOKEN\_DESPEDIDA}
	\item \footnotesize{TOKEN\_IDENTIFICACION}
	\item \footnotesize{TOKEN\_PROHIBIDA}
	\item \footnotesize{TOKEN\_SENTIMIENTO}
	\item \footnotesize{TOKEN\_DESCONOCIDO}
\end{enumerate}

Por lo tanto, si una palabra aparece en múltiples categorías, se prioriza el tipo más
específico.

\subsubsection{Solución de ambigüedades}
Para identificar frases en el tokenizador basado en hashmaps, se emplea un buffer de hasta 3
palabras, lo que permite dar prioridad a aquellas que sean mas largas. Si no se encuentra una
coincidencia con 3 palabras, se intenta con 2 y luego con 1.

En el caso del tokenizador AFD se utiliza el algoritmo de \textit{longest match}, es decir, se
continúa la lectura mientras el autómata permita avanzar hacia un estado válido, favoreciendo
la identificación de frases completas. Más detalles sobre el algoritmo de "longest match" y el
proceso de carga del AFD se explican en las secciones siguientes.

La resolucion de palabras o frases repetidas se realiza
mediante la carga del AFD en el siguiente orden:

\begin{verbatim}
self._cargar_frases_al_afd(self.tabla.palabras, TIPO_SENTIMIENTO)
self._cargar_frases_al_afd(self.tabla.palabras_prohibidas, TIPO_PROHIBIDA)
self._cargar_frases_al_afd(self.tabla.identificaciones, TIPO_IDENTIFICACION)
self._cargar_frases_al_afd(self.tabla.despedidas, TIPO_DESPEDIDA)
self._cargar_frases_al_afd(self.tabla.saludos, TIPO_SALUDO)
\end{verbatim}

Así, si una palabra aparece en más de un conjunto, se considera como "canónica" aquella
asociada al tipo más específico.
