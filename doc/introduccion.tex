% FIX: unificar resumen e introduccion, ademas de solucionar lo de las secciones.
\section*{\centering Resumen}
Este trabajo presenta un sistema para el análisis de interacciones en contact centers
utilizando técnicas de Speech Analytics y tokenización. El objetivo principal es procesar
transcripciones de audio, clasificando palabras en categorías como saludos, despedidas,
identificaciones y términos con connotación emocional, además de realizar análisis de
sentimientos. 

La implementación base utiliza un autómata finito determinista (AFD) para el reconocimiento de
patrones conversacionales y verificación de protocolos de atención. Como contribución adicional
derivada del tiempo disponible, hemos desarrollado una segunda versión intercambiable del
tokenizador basada en tablas hash, permitiendo comparar ambos enfoques en términos de
eficiencia y precisión.

\section{Introducción}

En el contexto de los contact centers, donde la calidad de la interacción es crucial,
presentamos un sistema con dos implementaciones modularizadas de tokenización.

El núcleo del sistema emplea un Autómata Finito Determinista (AFD) capaz de identificar
estructuras conversacionales complejas, como secuencias de saludo, verificar el cumplimiento de
protocolos de servicio y facilitar su extensión mediante la adición de nuevos estados.

Como desarrollo complementario, se implementó una versión intercambiable basada en diccionarios
hash, diseñada para búsquedas léxicas de tiempo constante y optimizada para el procesamiento
eficiente de términos sueltos. Esta versión comparte la misma interfaz que el tokenizador
basado en AFD, lo que permite alternar entre ambas sin alterar el resto del sistema.

Ambas implementaciones pueden seleccionarse en tiempo de ejecución, lo cual demuestra cómo un
mismo problema puede abordarse mediante paradigmas computacionales distintos, pero equivalentes
en funcionalidad. En todos los casos, el sistema conserva su capacidad para clasificar
semánticamente segmentos como saludos y despedidas, aplicar análisis de sentimientos mediante
puntuación léxica y generar métricas sobre la calidad del servicio.

Este diseño dual no solo enriquece el valor académico del proyecto, sino que además permite
obtener evidencia empírica sobre las ventajas comparativas de cada enfoque en distintos
escenarios operativos. Los detalles de implementación se presentan en la Sección 4.
