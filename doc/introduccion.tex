\section{Introducción}
Este trabajo presenta un sistema modular para el análisis de interacciones en contact centers,
combinando técnicas de Speech Analytics y tokenización. El sistema procesa transcripciones de
audio para clasificar palabras en categorías semánticas como saludos, despedidas,
identificaciones y términos con carga emocional, además de realizar análisis de sentimientos.

La implementación base utiliza un Autómata Finito Determinista (AFD) para reconocer patrones
conversacionales complejos y verificar protocolos de atención, con capacidad de extensión
mediante nuevos estados. Como contribución adicional, se desarrolló una versión alternativa
basada en tablas hash (técnicamente es una implementación híbrida dado su uso de expresiones
regulares para extraer palabras de signos de puntuación), optimizada para búsquedas léxicas en
tiempo constante y compatible con la misma interfaz del tokenizador basado en AFD. Esta
arquitectura dual permite comparar ambos enfoques en términos de eficiencia y precisión,
seleccionando la implementación deseada en tiempo de ejecución sin modificar el sistema. El
diseño no solo enriquece el valor académico del proyecto, sino que también proporciona
evidencia empírica sobre las ventajas de cada enfoque en distintos escenarios operativos.
