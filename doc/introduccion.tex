\section*{\centering Resumen}
Este trabajo presenta un sistema para el análisis de interacciones en contact centers
utilizando técnicas de Speech Analytics y tokenización. El objetivo es procesar el texto
derivado de las transcripciones de audio, clasificando las palabras en categorías como saludos,
despedidas, identificaciones, palabras prohibidas, palabras positivas y palabras negativas, así
como realizar un análisis de sentimientos basado en esas palabras. El sistema utiliza un
autómata finito determinista (AFD) para identificar los lexemas y estructurar el flujo de
interacción. Además, el AFD permite detectar y verificar el cumplimiento de un protocolo
estándar de atención por parte del agente. Finalmente, se presentan y analizan los resultados
obtenidos del procesamiento de textos simulados, destacando las conclusiones sobre la
efectividad del sistema en la mejora de la calidad de atención al cliente.

\section{Introducción}
El propósito de este trabajo práctico es implementar un sistema que procese las interacciones
comunes en un centro de contacto (contact center) y que, utilizando técnicas de tokenización y
análisis léxico, permita evaluar la calidad de la atención brindada por los agentes. El
objetivo final es identificar el sentimiento de las conversaciones y verificar si el protocolo
de atención al cliente ha sido seguido correctamente.

El sistema desarrollado se centra en la clasificación léxica de las palabras en categorías
específicas, como saludos, despedidas, identificaciones y palabras con connotaciones
emocionales (positivas y negativas). Este enfoque es esencial en el contexto de un contact
center, donde la precisión y la calidad de la comunicación son fundamentales para garantizar la
satisfacción del cliente. Para lograr esto, se implementa un Autómata Finito Determinista (AFD)
que facilita la identificación y estructuración de lexemas, permitiendo un mapeo efectivo de la
interacción entre el cliente y el agente.

El uso del AFD permite procesar las interacciones de manera eficiente y precisa, clasificando
los lexemas en tiempo real. Esta capacidad no solo optimiza el flujo de datos, sino que también
habilita la verificación del cumplimiento de un protocolo estándar de atención, asegurando que
se sigan las mejores prácticas durante la interacción. La estructura del AFD es flexible, lo
que permite la incorporación de nuevos términos y categorías a medida que evoluciona el
lenguaje utilizado en el servicio al cliente.

Además, el análisis de sentimientos se lleva a cabo mediante una tabla de puntuaciones
asignadas a las palabras identificadas. Este análisis permite evaluar la calidad de la
comunicación y la satisfacción del cliente de manera cuantitativa, proporcionando una visión
integral del estado emocional de la interacción. La capacidad de medir el tono y el sentimiento
detrás de las palabras ayuda a identificar patrones de comportamiento, lo que resulta
invaluable para la mejora continua de los procesos de atención al cliente. Los resultados
obtenidos del procesamiento de textos se presentan al final del análisis, destacando la
efectividad del sistema para identificar patrones de comportamiento en el desempeño del
servicio al cliente.
