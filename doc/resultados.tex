\section{Resultados}
El siguiente es el resultado del análisis de un ejemplo de conversación entre un agente y un
cliente. Se verifica que el agente ha cumplido con el protocolo de atención al cliente,
incluyendo saludo, identificación, y despedida, sin utilizar palabras prohibidas.

Además, las métricas y el análisis de sentimiento se realizan por separado para el cliente y el
agente, lo que facilita aún más la detección de discrepancias o valores atípicos en la
conversación.

\subsection{Caso de ejemplo}
\subsubsection*{Entrada:}
\begin{verbatim}
-----------------------------------------------------------------    
Agente: ¡Buenos días! Gracias por contactar con el servicio al 
cliente de ConexiónNet. Mi nombre es Johanna. ¿Cómo puedo 
ayudarle hoy?

Cliente: Hola, estoy teniendo algunos problemas con mi 
internet. Pago por 100 Mbps, pero últimamente he notado que 
la velocidad es mucho más baja.

Agente: Lamento escuchar eso, y estoy aquí para ayudarle 
a resolverlo. ¿Podría proporcionarme su nombre completo 
y el número de cuenta para que pueda revisar su situación?

Cliente: Claro, soy Juan Pérez y mi número de cuenta es 
12345678.

Agente: ¡Gracias, Juan! Voy a revisar su cuenta ahora mismo.
Veo que efectivamente tiene un plan de 100 Mbps. 
Déjeme hacer una prueba de línea para verificar cómo está 
funcionando su conexión en este momento.

Resto del texto ... 
-----------------------------------------------------------------    
\end{verbatim}

\subsubsection*{Salida:}
\begin{tcolorbox}[colback=gray!10, colframe=gray!80, sharp corners, boxrule=0.5pt]
	\begin{verbatim}
=== RESUMEN GENERAL ===
Puntaje total: -10
Sentimiento: NEGATIVO

=== CLIENTE ===
Puntaje: -6
Saludo: No
Despedida: Sí
Identificación: Sí
Palabras prohibidas: No

=== AGENTE ===
Puntaje: -4
Saludo: Sí
Despedida: Sí
Identificación: Sí
Palabras prohibidas: Sí

Nota: Se ignoraron las palabras desconocidas
          \end{verbatim}
\end{tcolorbox}

\subsubsection*{Tokens:}
\begin{tcolorbox}[colback=gray!10, colframe=gray!80, sharp corners, boxrule=0.5pt]
	\begin{verbatim}
	Valor                  Tipo                    Puntaje
	------------------------------------------------------
	agente:                TOKEN_AGENTE                  0
	Buenos días            TOKEN_SALUDO                  0
	,                      TOKEN_SIGNO_PUNTUACION        0
	¡                      TOKEN_SIGNO_PUNTUACION        0
	Gracias                TOKEN_SENTIMIENTO             1
	por                    TOKEN_DESCONOCIDO             0
	contactar              TOKEN_DESCONOCIDO             0
	con                    TOKEN_DESCONOCIDO             0
	el                     TOKEN_DESCONOCIDO             0

    Mas tokens...
\end{verbatim}
\end{tcolorbox}

\subsubsection*{AFD:}
Salida solo presente con el tokenizador de tipo AFD.
\begin{tcolorbox}[colback=gray!10, colframe=gray!80, sharp corners, boxrule=0.5pt]
    \begin{verbatim}
{
  "initial": "start",
  "states": {
    "start": {
      "transitions": {
        "e": [
          "start_e_sentimiento", 
          ...
          "start_e_saludo", 
          "start_e_despedida", 
          "start_e_prohibida"
        ],
        "m": [
          "start_m_sentimiento",
          ...
        ],
        ... mas transiciones
      },
      "is_final": true,
      "token_type": "prohibida",
      "puntuacion": -5
    },
    "start_e_sentimiento": {
      "transitions": {
        "x": [
          "start_e_sentimiento_x_sentimiento",
          ...
        ],
        ... mas transiciones
      },
      ... mas estados
    }
    ...
  }
}
\end{verbatim}
\end{tcolorbox}
