\section{Resultados}
El siguiente es el resultado del análisis de un ejemplo de conversación entre un agente y un
cliente, en el cual se detecta un sentimiento general positivo. Los tokens asociados a palabras
positivas superan en número y en puntaje a las palabras negativas, lo que sugiere una
interacción satisfactoria. Además, se verifica que el agente ha cumplido con el protocolo de
atención al cliente, incluyendo saludo, identificación, y despedida, sin utilizar palabras
prohibidas.

\subsection*{Caso de ejemplo}
\subsection*{Entrada:}
\begin{verbatim}
-----------------------------------------------------------------    
Agente: ¡Buenos días! Gracias por contactar con el servicio al 
cliente de ConexiónNet. Mi nombre es Johanna. ¿Cómo puedo 
ayudarle hoy?

Cliente: Hola, estoy teniendo algunos problemas con mi 
internet. Pago por 100 Mbps, pero últimamente he notado que 
la velocidad es mucho más baja.

Agente: Lamento escuchar eso, y estoy aquí para ayudarle 
a resolverlo. ¿Podría proporcionarme su nombre completo 
y el número de cuenta para que pueda revisar su situación?

Cliente: Claro, soy Juan Pérez y mi número de cuenta es 
12345678.

Agente: ¡Gracias, Juan! Voy a revisar su cuenta ahora mismo.
Veo que efectivamente tiene un plan de 100 Mbps. 
Déjeme hacer una prueba de línea para verificar cómo está 
funcionando su conexión en este momento.

Cliente: Sí, por favor. Me gustaría que se resolviera pronto.

Agente: ¡Entiendo perfectamente! Estoy realizando la prueba.
Vaya, parece que hay algunas fluctuaciones en la señal. 
No se preocupe, eso es algo que podemos solucionar. Voy a 
reiniciar su línea desde aquí, lo que puede ayudar a estabilizar 
la conexión. Esto tomará solo un minuto, y perderá la conexión 
brevemente. ¿Le parece bien?

Cliente: Sí, eso suena bien. Espero que funcione.

Agente: ¡Genial! Estoy reiniciando su línea. Listo. 
El reinicio se ha completado. Ahora, por favor, reinicie también 
su módem y router manualmente. Simplemente desconéctelos de la 
corriente durante 30 segundos y luego vuelva a conectarlos. Esto 
debería refrescar la señal.

Cliente: Está bien, lo haré. ¿Y si eso no soluciona el problema?

Agente: Si después de reiniciar su equipo el problema persiste, 
    no se preocupe. Programaremos una visita de un técnico a su 
    domicilio para asegurarnos de que todo esté en orden. Queremos 
    que disfrute del servicio por el que está pagando.

    Cliente: Eso suena bien. Solo quiero que funcione como debería.

    Agente: Lo comprendo completamente, Juan. Estoy segura de que 
    podremos resolver esto. Espero que el reinicio funcione, pero 
    si no, estaremos aquí para ayudarle. ¿Hay algo más en lo que 
    pueda asistirte hoy?

    Cliente: No, eso es todo. Solo espero que se solucione pronto.

    Agente: ¡Perfecto! Agradezco su paciencia y confianza. No dude 
    en llamarnos si necesita algo más. Le deseo un excelente día 
    y que su servicio se estabilice pronto. 
    ¡Estamos aquí para ayudarle!

    Cliente: Gracias, adiós.

    Agente: ¡Adiós, Juan! Que tenga un buen día.
    ----------------------------------------------------------------- 
    \end{verbatim}

\subsection*{Salida:}
\begin{tcolorbox}[colback=gray!10, colframe=gray!80, sharp corners, boxrule=0.5pt]
	\begin{verbatim}
    Sentimiento general: Positivo
    Puntaje total: 12
    Palabras positivas: ['bien', 'bien', 'genial', 'bien', 'bien', 
    'perfecto', 'paciencia', 'excelente']
    Palabras negativas: ['problema', 'problema']
    Verificación del protocolo de Atención (Agente):
        - Saludo: OK
          - Identificación: OK
          - Despedida: OK
          - Palabras rudas: Ninguna detectada
          \end{verbatim}
\end{tcolorbox}
