\documentclass[12pt,a4paper]{article}
\usepackage[utf8]{inputenc}
\usepackage{hyperref}
\usepackage[spanish, es-tabla]{babel}
\usepackage[version=3]{mhchem}
\usepackage[journal=jacs]{chemstyle}
\usepackage{amsmath}
\usepackage{amsfonts}
\usepackage{amssymb}
\usepackage{makeidx}
\usepackage[stable]{footmisc}
\usepackage[section]{placeins}
%%%% Entorno código - No usado de momento
\usepackage{tcolorbox}
\usepackage{listings}
\usepackage{xcolor}
\usepackage{verbatim}
%%% AFD GRAFO
\usepackage{tikz}
\usetikzlibrary{automata, positioning, arrows}

%Paquetes necesarios para imágenes, pies de página, etc.
\usepackage{graphicx}
\usepackage{lmodern}
\usepackage{fancyhdr}
% Margenes
%\usepackage[left=4cm,right=2cm,top=3cm,bottom=3cm]{geometry}
\usepackage[left=3cm,right=3cm,top=3cm,bottom=3cm]{geometry}
%% Algoritmos 
\usepackage{algorithm}
\usepackage{algorithmic} 

%Formato del título de las secciones

\usepackage{titlesec}
\usepackage{enumitem}
\titleformat*{\section}{\bfseries\large}
\titleformat*{\subsection}{\bfseries\normalsize}

%%%%%%%%%%%%%%%%%%%%%%
%Inicio del documento%
%%%%%%%%%%%%%%%%%%%%%%

\begin{document}
%%%%%%%%%%%%%%%%%%%%%%%%
%%%% CARATULA %%%%%%%%%%
%%%%%%%%%%%%%%%%%%%%%%5%
\begin{titlepage}
	\begin{center}
		\vspace*{1cm}

		\textbf{\LARGE{Análisis de Interacciones en Contact Centers con
				Speech Analytics y Tokenización}}

		\vspace{2cm}

		\textbf{\large{Alexis Rodrigo Arce Delgadillo}}

		\vfill

		\textbf{\large{Diseño de Compiladores}}

		\vspace{0.5cm}

		\textbf{\large{Profesor: Sergio Andrés Aranda Zemán}}

		\vspace{1.5cm}

		Noviembre 2024

		\vfill

	\end{center}
\end{titlepage}
%%%%%%%%%%%%%%%%%%%%%%%%%%%%%%%%%%%%%%%%%%%%%%%%

\vspace{7mm}

%%%%% INDICE %%%%%

\tableofcontents
\newpage

%%%%%%%%%%%%%%%%%%
\section*{\centering Resumen}
Este trabajo presenta un sistema para el análisis de interacciones en contact centers
utilizando técnicas de Speech Analytics y tokenización. El objetivo es procesar el texto
derivado de las transcripciones de audio, clasificando las palabras en categorías como saludos,
despedidas, identificaciones, palabras prohibidas, palabras positivas y palabras negativas, así
como realizar un análisis de sentimientos basado en esas palabras. El sistema utiliza un
autómata finito determinista (AFD) para identificar los lexemas y estructurar el flujo de
interacción. Además, el AFD permite detectar y verificar el cumplimiento de un protocolo
estándar de atención por parte del agente. Finalmente, se presentan y analizan los resultados
obtenidos del procesamiento de textos simulados, destacando las conclusiones sobre la
efectividad del sistema en la mejora de la calidad de atención al cliente.

\section{Introducción}
El propósito de este trabajo práctico es implementar un sistema que procese las interacciones
comunes en un centro de contacto (contact center) y que, utilizando técnicas de tokenización y
análisis léxico, permita evaluar la calidad de la atención brindada por los agentes. El
objetivo final es identificar el sentimiento de las conversaciones y verificar si el protocolo
de atención al cliente ha sido seguido correctamente.

El sistema desarrollado se centra en la clasificación léxica de las palabras en categorías
específicas, como saludos, despedidas, identificaciones y palabras con connotaciones
emocionales (positivas y negativas). Este enfoque es esencial en el contexto de un contact
center, donde la precisión y la calidad de la comunicación son fundamentales para garantizar la
satisfacción del cliente. Para lograr esto, se implementa un Autómata Finito Determinista (AFD)
que facilita la identificación y estructuración de lexemas, permitiendo un mapeo efectivo de la
interacción entre el cliente y el agente.

El uso del AFD permite procesar las interacciones de manera eficiente y precisa, clasificando
los lexemas en tiempo real. Esta capacidad no solo optimiza el flujo de datos, sino que también
habilita la verificación del cumplimiento de un protocolo estándar de atención, asegurando que
se sigan las mejores prácticas durante la interacción. La estructura del AFD es flexible, lo
que permite la incorporación de nuevos términos y categorías a medida que evoluciona el
lenguaje utilizado en el servicio al cliente.

Además, el análisis de sentimientos se lleva a cabo mediante una tabla de puntuaciones
asignadas a las palabras identificadas. Este análisis permite evaluar la calidad de la
comunicación y la satisfacción del cliente de manera cuantitativa, proporcionando una visión
integral del estado emocional de la interacción. La capacidad de medir el tono y el sentimiento
detrás de las palabras ayuda a identificar patrones de comportamiento, lo que resulta
invaluable para la mejora continua de los procesos de atención al cliente. Los resultados
obtenidos del procesamiento de textos se presentan al final del análisis, destacando la
efectividad del sistema para identificar patrones de comportamiento en el desempeño del
servicio al cliente.

\newpage
\section{Desarrollo}

\subsection{Descripción del algoritmo}
El algoritmo implementado es un sistema de análisis de interacciones en contact centers que
utiliza técnicas de tokenización y análisis de sentimientos. A continuación se detallan los
componentes y el funcionamiento del algoritmo:

\subsubsection{Autómata Finito Determinista (AFD)}
El AFD es la estructura central que permite la clasificación de las palabras en diferentes
categorías como saludos, identificaciones, despedidas, palabras positivas, negativas y
prohibidas. Se inicia con un estado inicial (q0) y tiene varios estados finales que
corresponden a las categorías mencionadas. Las transiciones se definen para cada palabra de
interés, lo que permite al AFD aceptar o rechazar lexemas basándose en las palabras que se
encuentran en el texto analizado.

\begin{figure}[h]
	\begin{minipage}{0.5\textwidth} % Primera columna (50% de la página)
		\centering
		\begin{tikzpicture}[scale=0.2]
			\tikzstyle{every node}+=[inner sep=0pt]
			\draw [black] (27.6,-7.2) circle (3);
			\draw (27.6,-7.2) node {$q_1$};
			\draw [black] (27.6,-7.2) circle (2.4);
			\draw [black] (18.7,-19.6) circle (3);
			\draw (18.7,-19.6) node {$q_0$};
			\draw [black] (35.4,-11.7) circle (3);
			\draw (35.4,-11.7) node {$q_2$};
			\draw [black] (35.4,-11.7) circle (2.4);
			\draw [black] (40.1,-17.9) circle (3);
			\draw (40.1,-17.9) node {$q_3$};
			\draw [black] (40.1,-17.9) circle (2.4);
			\draw [black] (40.1,-25.6) circle (3);
			\draw (40.1,-25.6) node {$q_4$};
			\draw [black] (40.1,-25.6) circle (2.4);
			\draw [black] (41.3,-32.8) circle (3);
			\draw (41.3,-32.8) node {$q_5$};
			\draw [black] (41.3,-32.8) circle (2.4);
			\draw [black] (33.7,-37.8) circle (3);
			\draw (33.7,-37.8) node {$q_6$};
			\draw [black] (33.7,-37.8) circle (2.4);
			\draw [black] (25.5,-39.6) circle (3);
			\draw (25.5,-39.6) node {$q_7$};
			\draw [black] (25.5,-39.6) circle (2.4);
			\draw [black] (10.2,-19.6) -- (15.7,-19.6);
			\fill [black] (15.7,-19.6) -- (14.9,-19.1) -- (14.9,-20.1);
			\draw [black] (19.79,-16.807) arc (155.24239:133.42043:24.973);
			\fill [black] (25.3,-9.13) -- (24.38,-9.31) -- (25.07,-10.04);
			\draw (21.59,-11.33) node [left] {$a$};
			\draw [black] (37.167,-18.528) arc (-79.13396:-91.78205:70.448);
			\fill [black] (37.17,-18.53) -- (36.29,-18.19) -- (36.48,-19.17);
			\draw (29.58,-20.14) node [below] {$c$};
			\draw [black] (21.214,-17.963) arc (121.57388:109.05947:57.481);
			\fill [black] (32.54,-12.61) -- (31.62,-12.39) -- (31.95,-13.34);
			\draw (25.74,-14.47) node [above] {$b$};
			\draw [black] (37.136,-25.14) arc (-99.88118:-111.44304:80.756);
			\fill [black] (37.14,-25.14) -- (36.43,-24.51) -- (36.26,-25.5);
			\draw (28.4,-23.9) node [below] {$d$};
			\draw [black] (38.331,-32.375) arc (-100.91962:-139.65628:31.083);
			\fill [black] (38.33,-32.37) -- (37.64,-31.73) -- (37.45,-32.71);
			\draw (27.6,-29.2) node [below] {$e$};
			\draw [black] (31.234,-36.093) arc (-126.9619:-154.04905:38.032);
			\fill [black] (31.23,-36.09) -- (30.9,-35.21) -- (30.29,-36.01);
			\draw (24.2,-31.33) node [left] {$f$};
			\draw [black] (23.398,-37.463) arc (-139.52535:-182.91859:21.267);
			\fill [black] (23.4,-37.46) -- (23.26,-36.53) -- (22.5,-37.18);
			\draw (18.68,-31.23) node [left] {$g$};
		\end{tikzpicture}
		\caption{Representación de la clase AFD}
		\label{fig:afd}
	\end{minipage}%
	\hfill
	\begin{minipage}{0.4\textwidth} % Segunda columna (50% de la página)
		\centering
		\begin{tabular}{|c|l|}
			\hline
			Símbolo & Representación - Tokens   \\
			\hline
			$q_0$   & INICIO                    \\
			$q_1$   & SALUDO                    \\
			$q_2$   & IDENTIFICACION            \\
			$q_3$   & DESPEDIDA                 \\
			$q_4$   & PALABRAS POSITIVAS        \\
			$q_5$   & PALABRAS NEGATIVAS        \\
			$q_6$   & PALABRAS PROHIBIDAS       \\
			$q_7$   & ERRORES LEXICOS           \\
			\hline
			Símbolo & Representación - Patrones \\
			\hline
			$a$     & Lista saludos             \\
			$b$     & Lista identificación      \\
			$c$     & Lista despedidas          \\
			$d$     & Lista palabras positivas  \\
			$e$     & Lista palabras negativas  \\
			$f$     & Lista palabras prohibidas \\
			$g$     & otras palabras            \\
			\hline
		\end{tabular}
	\end{minipage}
\end{figure}

\subsubsection{Tokenización}
El texto se procesa para extraer los lexemas, que son palabras individuales. Cada lexema se
evalúa utilizando el AFD; si el lexema coincide con una de las transiciones definidas, se
acepta y se cuenta en la categoría correspondiente. Si no, se considera un error léxico.

\subsubsection{Análisis de Sentimiento}
Se utiliza una tabla de sentimientos que asigna puntajes a palabras positivas y negativas. Al
procesar los lexemas, se acumula un puntaje total que determina si el sentimiento general de la
interacción es positivo, negativo o neutral.

\subsubsection{Verificación del Protocolo de Atención}
Se verifica si se cumplen ciertas condiciones de atención al cliente, como la presencia de
saludos y despedidas, y se detectan palabras que no son apropiadas.

\subsubsection{Estructura de Datos}
El AFD tiene un diccionario para almacenar las transiciones entre estados y otro para los
identificadores de lexemas. Un contador permite llevar el registro de cuántas veces se han
reconocido las palabras de cada categoría. La tabla de sentimientos utiliza una lista enlazada
donde es posible agregar o eliminar palabras y ponderaciones mediante el menú del sistema.

\subsubsection{Salida de Resultados}
Los resultados del análisis, que incluyen el sentimiento general, puntajes, palabras
clasificadas, y la verificación del protocolo, se guardan en un archivo de texto y en un
archivo JSON que representa el estado del AFD.

\subsection{Decisiones Tomadas}

\subsubsection{Selección de un AFD}
Se decidió implementar un Autómata Finito Determinista (AFD) como estructura principal para la
clasificación de palabras debido a su capacidad para manejar eficientemente un conjunto finito
de estados y transiciones. Esta elección permite un análisis rápido de las entradas de texto y
una categorización clara de los lexemas.

\subsubsection{Definición de Estados y Transiciones}
Se definieron múltiples estados finales en el AFD, cada uno asociado a distintas categorías de
palabras: saludos, identificaciones, despedidas, palabras prohibidas, palabras positivas y
palabras negativas. Esta organización estructurada permite un análisis claro y eficiente de las
interacciones con los clientes. Las palabras correspondientes a cada categoría se cargan
previamente desde archivos de texto, listos para su uso en el análisis.

Las transiciones del autómata se implementaron para reconocer un conjunto específico de
palabras, seleccionadas cuidadosamente para representar las categorías relevantes dentro del
contexto de atención al cliente y la verificación del protocolo que debe seguir el agente en
diversas situaciones.

\subsubsection{Implementación de la Tokenización}
Se optó por utilizar expresiones regulares para extraer lexemas del texto de entrada. Este
enfoque permite una identificación precisa de las palabras individuales y su procesamiento en
el AFD. Se incorporó un manejo de errores léxicos mediante la creación de un estado
\textbf{q\_ERROR\_LX} para capturar palabras no reconocidas. Esto mejora la robustez del
algoritmo al permitir una respuesta adecuada ante entradas inesperadas.

\subsubsection{Verificación del Protocolo}
Se implementó un mecanismo para verificar el cumplimiento del protocolo de atención al cliente,
asegurando que las interacciones contengan elementos clave como saludos y despedidas. Esto
ayuda a evaluar la calidad del servicio proporcionado por el agente. Esta decisión se tomó con
el objetivo de garantizar que el análisis no solo sea técnico, sino también relevante para las
prácticas de atención al cliente.

\subsubsection{Salida y Visualización de Resultados}
Se decidió que los resultados del análisis se guardarían tanto en un archivo de texto como en
un archivo JSON. Esto permite no solo una fácil revisión por parte de los usuarios, sino
también la posibilidad de almacenar el estado del AFD para futuros análisis. La estructura JSON
se eligió para facilitar la visualización y el análisis posterior del comportamiento del AFD y
el funcionamiento de las transiciones.

\subsection{Alcance del Proyecto}
El alcance del proyecto se puede definir a través de los siguientes puntos clave:

\subsubsection{Clasificación de Lexemas}
El sistema identifica y clasifica lexemas en diferentes categorías, permitiendo una mejor
comprensión del contexto de la interacción. Esto incluye clasificaciones específicas para
saludos, identificaciones, despedidas, palabras prohibidas y palabras que expresan
sentimientos.

\subsubsection{Análisis de Sentimientos}
Se realiza un análisis de sentimientos sobre las interacciones, evaluando las palabras
utilizadas para determinar si la conversación fue positiva, negativa o neutral. Esto se realiza
mediante la tabla de sentimientos donde cada palabra posee una ponderación pre-cargada desde un
archivo de texto.

\subsubsection{Verificación de Protocolos de Atención}
El sistema verifica si se cumplen los protocolos de atención al cliente, asegurando que las
interacciones incluyan los elementos necesarios para una atención efectiva. Esto incluye la
detección de saludos y despedidas apropiadas por parte del agente.

\subsubsection{Manejo de palabras no clasificadas}
La implementación del estado q\_ERROR\_LX permite manejar palabras no reconocidas o errores
léxicos, mejorando la robustez del sistema ante entradas inesperadas.

\subsubsection{Generación de Informes}
Los resultados del análisis se generan en archivos de texto donde se imprimen los reportes, una
tabla de lexemas y un JSON con las transiciones de cada palabra analizada, lo que permite una
fácil revisión.

\subsection{Estrategias Aplicadas}
Las estrategias aplicadas en el desarrollo del sistema incluyen:

\subsubsection{Desarrollo Modular}
El sistema se construyó de manera modular, separando las diferentes funcionalidades en clases y
métodos. Esto permite un mantenimiento más sencillo y la posibilidad de añadir nuevas
características en el futuro.

\subsubsection{Uso de AFD para Eficiencia}
La elección de un AFD como base del sistema permite una evaluación rápida y eficiente de los
lexemas, facilitando el procesamiento de grandes volúmenes de texto de interacciones.

\subsubsection{Implementación de Tablas de Sentimientos}
Se diseñó una tabla de sentimientos que asigna puntajes específicos a palabras según su
connotación, lo cual permite un análisis más detallado y preciso de las emociones expresadas en
las interacciones. Estos datos, pre-cargados desde un archivo de texto para garantizar su
persistencia, incluyen la flexibilidad de agregar o eliminar palabras según se requiera,
facilitando así la personalización y actualización continua del análisis de sentimiento.

\subsubsection{Recopilación de Datos y Mejora Continua}
Se incorporó un enfoque que permite la adición de nuevas palabras a las categorías existentes,
lo que permite que el sistema evolucione a medida que se identifican nuevas necesidades esto
mediante el uso de documentos de texto donde las palabras son guardadas.

\subsubsection{Validación de Resultados}
Se implementaron mecanismos para validar los resultados del sistema, asegurando que los
análisis sean precisos y útiles para evaluar la calidad del servicio de atención al cliente.

\newpage
\subsection{Código Fuente}

En esta sección, se presenta el código fuente del sistema desarrollado para el análisis de
interacciones en contact centers. El código está organizado en módulos, clases y métodos que
permiten la tokenización de texto, el análisis de sentimientos y la verificación de protocolos
de atención.

El sistema está estructurado de la siguiente manera:
\begin{tcolorbox}[colback=gray!10, colframe=gray!80, sharp corners, boxrule=0.5pt]
	\begin{verbatim}
    +-- audios
    |   +-- audio-texto.py
    |   +-- prueba-texto1-negativo.m4a
    |   +-- prueba-texto2-positivo.m4a
    +-- data
    |   +-- despedidas.txt
    |   +-- identificaciones.txt
    |   +-- palabras_prohibidas.txt
    |   +-- palabras_y_puntajes.txt
    |   +-- saludos.txt
    |
    +-- doc
    |   +-- ensayo.pdf
    |
    +-- models
    |   +-- afd.py
    |   +-- analisis.py
    |   +-- file_utils.py
    |   +-- menu.py
    |   +-- sentimientos.py
    |   +-- __init__.py
    |
    +-- output
    |   +-- afd.json
    |   +-- reporte.txt
    |   +-- tabla_lexemas.txt
    |
    +-- test
    |    +-- prueba-texto1-negativo.txt
    |    +-- prueba-texto2-positivo.txt
    |    +-- prueba-texto3-positivo.txt
    |    +-- prueba-texto4-neutral.txt
    |    +-- prueba.txt
    |
    |- tokenizador.py
    |- README.md

\end{verbatim}
\end{tcolorbox}
\newpage
\subsubsection{Clase AFD}

\begin{verbatim}
import re
import json
import os

class AFD:
    def __init__(self, estado_inicial, estados_finales):
        self.estado_inicial = estado_inicial
        self.estados_finales = estados_finales
        self.transiciones = {estado_inicial: {}}
        self.lexema_ids = {}
        self.next_id = 1
        self.transiciones_usadas = {}  
        
    def agregar_transicion(self,estado_origen,lexema,estado_destino):
        if estado_origen not in self.transiciones:
            self.transiciones[estado_origen] = {}
        if lexema in self.transiciones[estado_origen]:
            raise Exception("Transicion duplicada")
        self.transiciones[estado_origen][lexema] = estado_destino
        
        if lexema not in self.lexema_ids:
            self.lexema_ids[lexema] = self.next_id
            self.next_id += 1
        return self.lexema_ids[lexema]


    def evaluar_lexema(self, lexema):
        estado_actual = self.estado_inicial
        if lexema in self.transiciones[estado_actual]:
            estado_actual = self.transiciones[estado_actual][lexema]
        else:
            return estado_actual, False

        return estado_actual, True
    def vaciar(self):
        self.transiciones = {self.estado_inicial: {}}
        self.lexema_ids.clear()  
        self.next_id = 1
        self.transiciones_usadas.clear()    
    def get_ids(self):
        return self.lexema_ids

    def guardar_en_json(self, filename):
        data = {
            'estado_inicial': self.estado_inicial,
            'estados_finales': self.estados_finales,
            'transiciones': {self.estado_inicial:{}}
        }
        for estado, transiciones in self.transiciones_usadas.items():
            for transicion in transiciones:
                data['transiciones'][self.estado_inicial][transicion] = estado  
                                    
        with open(os.path.join("output",filename), 'w', encoding='utf-8') as f:
            json.dump(data, f, ensure_ascii=False, indent=4)
        
    def generar_tabla_lexemas(self):
        tabla = []
        for estado, lexemas in self.transiciones_usadas.items():
            for lexema in lexemas:
                patrones = [lexema 
                    for lexema in self.transiciones[self.estado_inicial] 
                    if self.transiciones[self.estado_inicial][lexema] == estado]
                tabla.append({
                    'Lexema': lexema,
                    'Token': estado,
                    'Patron': patrones
                })
        return tabla
\end{verbatim}

\section*{Explicación de la Clase AFD}
\begin{quote}
	\textbf{Inicialización:} La clase \texttt{AFD} se inicializa definiendo el estado inicial, los estados finales y las transiciones que definen el flujo del autómata.\\
	\textbf{Métodos:}
	\begin{itemize}
		\item \textbf{agregar\_transicion:} Permite agregar transiciones entre estados mediante lexemas, asegurando que no haya duplicados en las transiciones.
		\item \textbf{evaluar\_lexema:} Evalúa si un lexema puede ser procesado desde el estado inicial hacia uno de los estados finales, indicando si se encontró una transición válida.
		\item \textbf{vaciar:} Reinicia el AFD, eliminando todas las transiciones y reiniciando los IDs de lexemas.
		\item \textbf{guardar\_en\_json:} Guarda el estado y las transiciones del autómata en un archivo JSON.
		\item \textbf{generar\_tabla\_lexemas:} Genera una tabla organizada de lexemas, mostrando su token y patrón asociado, para facilitar el análisis de la clasificación de palabras.
	\end{itemize}
\end{quote}


\subsubsection{Clase TokenizadorAFD}
\begin{verbatim}
class TokenizadorAFD:
    def __init__(self):
        self.afd = self.construir_afd()
        self.lista_salida = []
        self.posicion = 0
    def cargar_palabras_y_puntajes(self, archivo):
        palabras_positivas = []
        palabras_negativas = []
        with open(os.path.join("sentiment_symbols", archivo), "r", encoding="utf-8") as f:
            for linea in f:
                palabra, puntaje = linea.strip().split(",")
                puntaje = int(puntaje)
                if puntaje > 0:
                    palabras_positivas.append(palabra)
                else:
                    palabras_negativas.append(palabra)
            f.close()
        return palabras_positivas, palabras_negativas
    def cargar_palabras(self,archivo):
        lexemas = []
        with open(os.path.join("sentiment_symbols",archivo),"r",encoding="utf-8") as f:
            for linea in f:
                lexemas.append(linea.strip())
        return lexemas

    def construir_afd(self):
        estado_inicial = "q0"
        estado_finales = {
            'SALUDO': 'q_SALUDO',
            'IDENTIFICACION': 'q_IDENTIFICACION',
            'DESPEDIDA': 'q_DESPEDIDA',
            'ERROR_LX': 'q_ERROR_LX',
            'POSITIVO': 'q_POSITIVO',
            'NEGATIVO': 'q_NEGATIVO',
            'PALABRA_PROHIBIDA': 'q_PALABRA_PROHIBIDA' 

        }     
        afd = AFD(estado_inicial, estado_finales)
        
        saludos = self.cargar_palabras("saludos.txt")
        identificaciones = self.cargar_palabras("identificaciones.txt")
        despedidas = self.cargar_palabras("despedidas.txt")
        palabras_prohibidas = self.cargar_palabras("palabras_prohibidas.txt")
        palabras_positivas, palabras_negativas = 
                    self.cargar_palabras_y_puntajes("palabras_y_puntajes.txt")

        for saludo in saludos:
            afd.agregar_transicion(estado_inicial, saludo, "q_SALUDO")

        for identificacion in identificaciones:
            afd.agregar_transicion(estado_inicial,
                                 identificacion, "q_IDENTIFICACION")

        for despedida in despedidas:
            afd.agregar_transicion(estado_inicial, despedida, 
                                "q_DESPEDIDA")

        for palabra_positiva in palabras_positivas:
            afd.agregar_transicion(estado_inicial, palabra_positiva, 
                                "q_POSITIVO")

        for palabra_negativa in palabras_negativas:
            afd.agregar_transicion(estado_inicial, palabra_negativa, 
                                "q_NEGATIVO")
            
        for palabra_prohibida in palabras_prohibidas:
            afd.agregar_transicion(estado_inicial, palabra_prohibida, 
                                "q_PALABRA_PROHIBIDA")
        
        return afd


    def tokenizador(self, texto):
        lexemas = re.findall(r'\b\w+:|\b\w+', texto.lower())  
        
        for lexema in lexemas:
            estado, es_aceptado = self.afd.evaluar_lexema(lexema)
            
            if es_aceptado:
                print(f"Lexema aceptado: [{lexema}] se detecto como {estado}")
                self.posicion += 1
                
                if estado not in self.afd.transiciones_usadas:
                    self.afd.transiciones_usadas[estado] = []
                self.afd.transiciones_usadas[estado].append(lexema)

            else:
                print(f"Lexema no aceptado: [{lexema}]")
                self.agregar_error_lexico(lexema)
                self.posicion += 1
                
                error_estado = "q_ERROR_LX"
                if error_estado not in self.afd.transiciones_usadas:
                    self.afd.transiciones_usadas[error_estado] = []
                self.afd.transiciones_usadas[error_estado].append(lexema)
        return lexemas


    def agregar_error_lexico(self, error_lx):
        self.afd.agregar_transicion("q0", error_lx, "q_ERROR_LX")

\end{verbatim}

\section*{Explicación de la Clase TokenizadorAFD}
\begin{quote}
	\textbf{Inicialización:} La clase \texttt{TokenizadorAFD} se inicializa creando el autómata finito determinista (AFD) y configurando la lista de salida y la posición actual para el procesamiento de tokens.

	\textbf{Métodos:}
	\begin{itemize}
		\item \textbf{cargar\_palabras\_y\_puntajes:} Carga palabras junto con sus puntajes desde un archivo, organizándolas en listas de palabras positivas y negativas para facilitar su uso en el AFD.

		\item \textbf{cargar\_palabras:} Carga un conjunto de lexemas desde un archivo de texto, permitiendo agrupar palabras relacionadas, como saludos o identificaciones, en listas para ser utilizadas en el autómata.

		\item \textbf{construir\_afd:} Define el AFD inicializando el estado de partida y los estados finales. Además, agrega las transiciones de los estados, asociando palabras como saludos, identificaciones, despedidas, palabras prohibidas y términos con connotaciones positivas o negativas a sus respectivos estados finales.

		\item \textbf{tokenizador:} Toma el texto de entrada y lo divide en lexemas utilizando expresiones regulares. Luego, para cada lexema, verifica si existe una transición definida en el AFD. Si se encuentra una transición, el lexema es clasificado en el estado correspondiente; si no, se trata como un error léxico y se registra en el estado de error. Este proceso también actualiza la posición del token para un análisis secuencial.

		\item \textbf{agregar\_error\_lexico:} Registra un lexema no reconocido en el estado de error del AFD, permitiendo una mejor identificación y manejo de palabras no válidas en el contexto de análisis léxico.

		\item \textbf{guardar\_en\_json:} Este método, aunque no se muestra en el código, típicamente permite guardar la configuración actual del AFD en un archivo JSON, lo cual facilita la observación de estados y transiciones entre ejecuciones.

	\end{itemize}
\end{quote}

\subsubsection{ Análisis de Sentimientos y Verificación de Protocolo}
\begin{verbatim}
import re
def analizar_sentimiento(tokens, tabla_sentimientos, tokenizador):
    puntaje_total = 0
    palabras_positivas = []
    palabras_negativas = []

    palabras_rudas_detectadas = False

    for token in tokens:
        estado, es_aceptado = tokenizador.afd.evaluar_lexema(token)
        if es_aceptado and estado == 'q_PALABRA_PROHIBIDA':
            palabras_rudas_detectadas = True
            break 

    if palabras_rudas_detectadas:
        return {
            "sentimiento": "Negativo",
            "puntaje_total": -1,
            "palabras_positivas": [],
            "palabras_negativas": []
        }
    
    for token in tokens:
        puntaje = tabla_sentimientos.obtener_puntaje(token)
        puntaje_total += puntaje

        if puntaje > 0:
            palabras_positivas.append(token)
        elif puntaje < 0:
            palabras_negativas.append(token)
    
    sentimiento = "Positivo" if puntaje_total > 0 
        else "Negativo" if puntaje_total < 0 else "Neutral"
    
    return {
        "sentimiento": sentimiento,
        "puntaje_total": puntaje_total,
        "palabras_positivas": palabras_positivas,
        "palabras_negativas": palabras_negativas
    }

def verificar_protocolo(lexemas, tokenizador):
    resultados = {
        "saludo": False,
        "identificacion": False,
        "despedida": False,
        "palabras_prohibidas": False
    }

    analizando_agente = False
    mensaje_agente = []

    for lexema in lexemas:
        if "agente:" in lexema:
            analizando_agente = True
            mensaje_agente.append(lexema.split("agente:", 1)[1].strip())
        elif "cliente:" in lexema and analizando_agente:
            analizando_agente = False
            procesar_mensaje_agente(mensaje_agente, tokenizador, resultados)
            mensaje_agente = []  

        elif analizando_agente:
            mensaje_agente.append(lexema)

    if mensaje_agente:
        procesar_mensaje_agente(mensaje_agente, tokenizador, resultados)

    return resultados

def procesar_mensaje_agente(mensaje_agente, tokenizador, resultados):
    mensaje = ' '.join(mensaje_agente)
    tokens = re.findall(r'\b\w+\b', mensaje.lower())

    for token in tokens:
        estado,es_aceptado = tokenizador.afd.evaluar_lexema(token)

        if es_aceptado:
            if estado == 'q_SALUDO':
                resultados["saludo"] = True
            elif estado == 'q_IDENTIFICACION':
                resultados["identificacion"] = True
            elif estado == 'q_DESPEDIDA':
                resultados["despedida"] = True
            elif estado == 'q_PALABRA_PROHIBIDA':
                resultados["palabras_prohibidas"] = True

\end{verbatim}
\section*{Funciones de Análisis de Sentimiento y Verificación de Protocolo}

\textbf{Función \texttt{analizar\_sentimiento}}\\
La función \texttt{analizar\_sentimiento} clasifica el sentimiento de un conjunto de lexemas (palabras) y detecta la presencia de palabras prohibidas.

\textbf{Entradas:}
\begin{itemize}
	\item \texttt{lexemas}: Lista de palabras de la conversación.
	\item \texttt{tabla\_sentimientos}: Objeto para obtener el puntaje asociado a cada palabra.
	\item \texttt{tokenizador}: Instancia de \texttt{TokenizadorAFD} que permite identificar el estado asociado a cada lexema.
\end{itemize}

\textbf{Proceso:}
\begin{enumerate}
	\item Se revisa cada lexema para verificar si pertenece al estado \texttt{q\_PALABRA\_PROHIBIDA}, indicando la detección de una palabra prohibida y terminando el análisis de inmediato con un sentimiento “Negativo”.
	\item Si no hay palabras prohibidas, se calcula el \texttt{puntaje\_total} sumando los puntajes asociados a cada lexema: las palabras con puntaje positivo se agregan a \texttt{palabras\_positivas} y las de puntaje negativo a \texttt{palabras\_negativas}.
	\item Según el \texttt{puntaje\_total}, el sentimiento resultante es:
	      \begin{itemize}
		      \item “Positivo” si el puntaje es mayor a 0.
		      \item “Negativo” si el puntaje es menor a 0.
		      \item “Neutral” si el puntaje es igual a 0.
	      \end{itemize}
\end{enumerate}

\textbf{Salida:} Un diccionario con el sentimiento, puntaje total y listas de palabras positivas y negativas.

\textbf{Función \texttt{verificar\_protocolo}}\\
Esta función verifica si el agente sigue el protocolo mediante la detección de un saludo, identificación, despedida y ausencia de palabras prohibidas.

\textbf{Entradas:}
\begin{itemize}
	\item \texttt{lexemas}: Lista de palabras extraídas de la interacción.
	\item \texttt{tokenizador}: Instancia de \texttt{TokenizadorAFD} que permite identificar el estado de cada lexema.
\end{itemize}

\textbf{Proceso:}
\begin{enumerate}
	\item Se separan los mensajes del agente y del cliente en el texto.
	\item Para cada segmento del agente, se llama a la función \texttt{procesar\_mensaje\_agente}.
\end{enumerate}

\textbf{Función \texttt{procesar\_mensaje\_agente}}\\
Esta función procesa cada mensaje del agente y clasifica cada lexema en los estados de protocolo esperados.

\textbf{Entradas:}
\begin{itemize}
	\item \texttt{mensaje\_agente}: Lista de lexemas de los mensajes del agente.
	\item \texttt{tokenizador}: Instancia del \texttt{TokenizadorAFD}.
	\item \texttt{resultados}: Diccionario que almacena el estado del protocolo (saludo, identificación, despedida y palabras prohibidas).
\end{itemize}

\textbf{Proceso:}
\begin{enumerate}
	\item Combina los mensajes del agente en una cadena y la tokeniza.
	\item Revisa cada lexema para determinar si pertenece a alguno de los estados de protocolo (\texttt{q\_SALUDO}, \texttt{q\_IDENTIFICACION}, \texttt{q\_DESPEDIDA}, \texttt{q\_PALABRA\_PROHIBIDA}).
	\item Actualiza el diccionario \texttt{resultados} según el estado detectado.
\end{enumerate}

\textbf{Salida:} El diccionario \texttt{resultados} actualizado.
\subsubsection{Clase Manejo de Tabla de Sentimientos}
\begin{verbatim}
    class NodoSentimiento:
    def __init__(self, palabra, puntaje):
        self.palabra = palabra
        self.puntaje = puntaje
        self.siguiente = None

class TablaSentimientos:
    def __init__(self):
        self.cabeza = None

    def agregar_palabra(self, palabra, puntaje):
        nuevo_nodo = NodoSentimiento(palabra, puntaje)
        if self.cabeza is None:
            self.cabeza = nuevo_nodo
        else:
            actual = self.cabeza
            while actual.siguiente:
                actual = actual.siguiente
            actual.siguiente = nuevo_nodo

    def obtener_puntaje(self, palabra):
        actual = self.cabeza
        while actual:
            if actual.palabra == palabra:
                return actual.puntaje
            actual = actual.siguiente
        return 0  

    def eliminar_palabra(self, palabra):
        actual = self.cabeza
        anterior = None
        while actual:
            if actual.palabra == palabra:
                if anterior is None: 
                    self.cabeza = actual.siguiente
                else:
                    anterior.siguiente = actual.siguiente
                return  
            anterior = actual
            actual = actual.siguiente
        print(f"La palabra '{palabra}' no se encontró 
        en la tabla de sentimientos.")

\end{verbatim}
\section*{Estructura de la Tabla de Sentimientos}

La estructura de la \texttt{TablaSentimientos} y la clase \texttt{NodoSentimiento} permiten
almacenar y gestionar palabras asociadas a un puntaje de sentimiento en una lista enlazada.
Esta estructura se utiliza para calcular el puntaje de sentimiento de un conjunto de palabras.

\subsection*{Clase \texttt{NodoSentimiento}}
La clase \texttt{NodoSentimiento} representa un nodo en la lista enlazada y contiene los atributos:
\begin{itemize}
	\item \texttt{palabra}: La palabra asociada al sentimiento.
	\item \texttt{puntaje}: El puntaje de sentimiento de la palabra, que puede ser positivo, negativo o cero.
	\item \texttt{siguiente}: Referencia al siguiente nodo en la lista enlazada.
\end{itemize}

\subsection*{Clase \texttt{TablaSentimientos}}
La clase \texttt{TablaSentimientos} contiene métodos para gestionar la lista de palabras con sus puntajes. Sus métodos principales son:

\begin{itemize}
	\item \texttt{agregar\_palabra(palabra, puntaje)}:
	      Este método agrega un nuevo nodo al final de la lista enlazada con la palabra y puntaje especificados. Si la lista está vacía, el nodo se convierte en la cabeza.

	\item \texttt{obtener\_puntaje(palabra)}:
	      Busca en la lista la palabra proporcionada y devuelve el puntaje asociado. En caso de no encontrar la palabra, retorna 0. Este método recorre la lista desde la cabeza hasta el final.

	\item \texttt{eliminar\_palabra(palabra)}:
	      Elimina el nodo que contiene la palabra especificada. Para ello:
	      \begin{itemize}
		      \item Si el nodo a eliminar es la cabeza, la cabeza se actualiza para que apunte al siguiente nodo.
		      \item Si la palabra se encuentra en otro nodo, se enlaza el nodo anterior al siguiente nodo del actual.
	      \end{itemize}
	      Si la palabra no se encuentra, se muestra un mensaje indicando que no existe en la tabla.
\end{itemize}

\subsection*{Uso de la Tabla de Sentimientos}
Esta estructura permite almacenar y recuperar palabras con sus puntajes de manera eficiente,
facilitando el análisis de sentimientos en otras funciones del programa mediante métodos de
inserción, búsqueda y eliminación en una lista enlazada.

\subsubsection{Gestión de Sentimientos}

\begin{verbatim}
    import os
    from tokenizer.sentimientos import TablaSentimientos
    
    def eliminar_archivo(nombre_archivo):
        try:
            os.remove(os.path.join("output",nombre_archivo))
        except FileNotFoundError:
            pass
        except Exception as e:
            pass
    def cargar_palabras_y_puntajes(archivo):
        tabla_sentimientos = TablaSentimientos()
        with open(os.path.join("sentiment_symbols", archivo), "r", encoding="utf-8") as f:
            for linea in f:
                palabra, puntaje = linea.strip().split(",")
                tabla_sentimientos.agregar_palabra(palabra, int(puntaje))
            f.close()    
        return tabla_sentimientos
    
    def agregar_palabra_sentimiento(tabla_sentimientos, afd):
        palabra = input("Ingrese la nueva palabra: ").strip()
        puntaje = int(input("Ingrese el puntaje de la palabra 
        (positivo o negativo): "))
        eliminar_archivo("tabla_lexemas.txt")
        eliminar_archivo("afd.json")    
        tabla_sentimientos.agregar_palabra(palabra, puntaje)        
        afd.agregar_transicion(afd.estado_inicial, palabra, 
        f"q_{puntaje > 0 and 'POSITIVO' or 'NEGATIVO'}")        
        with open(os.path.join("sentiment_symbols",
        "palabras_y_puntajes.txt"), "a", encoding="utf-8") as f:
            f.write(f"{palabra},{puntaje}\n")
        
        print(f"Palabra '{palabra}' con puntaje {puntaje} añadida con éxito.")
    
    def eliminar_palabra_sentimiento(tabla_sentimientos):
        palabra = input("Ingrese la palabra a eliminar: ").strip()
        tabla_sentimientos.eliminar_palabra(palabra)
        eliminar_archivo("tabla_lexemas.txt")
        eliminar_archivo("afd.json")
        with open(os.path.join("sentiment_symbols",
        "palabras_y_puntajes.txt"), "r", encoding="utf-8") as f:
            lineas = f.readlines()
        
        with open(os.path.join("sentiment_symbols"
        "palabras_y_puntajes.txt"), "w", encoding="utf-8") as f:
            for linea in lineas:
                if not linea.startswith(f"{palabra},"):
                    f.write(linea)
    
        print(f"Palabra '{palabra}' eliminada con éxito.")
        
\end{verbatim}
\section*{Módulo de Gestión de Sentimientos}

Este módulo permite la carga, adición y eliminación de palabras con sus puntajes de sentimiento
en una \texttt{TablaSentimientos}, así como la actualización de un Autómata Finito Determinista
(AFD) para clasificar palabras. También se realiza la persistencia de datos en archivos.

\subsection*{Funciones Principales}

\subsubsection*{\texttt{eliminar\_archivo(nombre\_archivo)}}
Elimina un archivo en la carpeta \texttt{output}, ignorando errores si el archivo no existe o si ocurre otra excepción.
\begin{verbatim}
try:
    os.remove(os.path.join("output", nombre_archivo))
except FileNotFoundError:
    pass
except Exception as e:
    pass
\end{verbatim}

\subsubsection*{\texttt{cargar\_palabras\_y\_puntajes(archivo)}}
Carga palabras y puntajes desde un archivo en la carpeta \texttt{data}, creando una instancia
de \texttt{TablaSentimientos}. El archivo debe contener líneas en el formato
\texttt{palabra,puntaje}. La función retorna la tabla de sentimientos.

\subsubsection*{\texttt{agregar\_palabra\_sentimiento(tabla\_sentimientos, afd)}}

Esta función permite agregar una nueva palabra con su puntaje de sentimiento a la
\texttt{TablaSentimientos} y al AFD. Realiza los siguientes pasos:
\begin{itemize}
	\item Elimina los archivos \texttt{tabla\_lexemas.txt} y \texttt{afd.json} si existen, para evitar conflictos con la actualización.
	\item Añade la palabra a la tabla de sentimientos con el puntaje especificado.
	\item Añade una transición al AFD en su estado inicial, con un nuevo estado basado en si el puntaje es positivo o negativo.
	\item Guarda la palabra y su puntaje en el archivo \texttt{palabras\_y\_puntajes.txt}.
\end{itemize}
Al finalizar, muestra un mensaje confirmando la adición exitosa de la palabra.

\subsubsection*{\texttt{eliminar\_palabra\_sentimiento(tabla\_sentimientos)}}
Permite eliminar una palabra de la \texttt{TablaSentimientos} y del archivo de persistencia:
\begin{itemize}
	\item Elimina los archivos \texttt{tabla\_lexemas.txt} y \texttt{afd.json} para forzar una actualización.
	\item Busca y elimina la palabra en el archivo \texttt{palabras\_y\_puntajes.txt}, escribiendo las líneas que no corresponden a la palabra eliminada.
\end{itemize}

Finalmente, muestra un mensaje confirmando la eliminación exitosa de la palabra.

\subsubsection{Función principal}

\begin{verbatim}
import os
from tokenizer.afd import TokenizadorAFD
from tokenizer.file_utils 
import agregar_palabra_sentimiento, eliminar_archivo,
    cargar_palabras_y_puntajes, eliminar_palabra_sentimiento

from tokenizer.menu import mostrar_menu
from tokenizer.analisis import analizar_sentimiento, verificar_protocolo


def main():
    tokenizador = TokenizadorAFD()
    tabla_sentimientos = 
            cargar_palabras_y_puntajes("palabras_y_puntajes.txt")

    while True:
        opcion = mostrar_menu()

        if opcion == '1':
            tokenizador.afd.vaciar() 
            tokenizador = TokenizadorAFD()
            tabla_sentimientos =
             cargar_palabras_y_puntajes("palabras_y_puntajes.txt")
            eliminar_archivo("tabla_lexemas.txt")
            eliminar_archivo("afd.json")
            eliminar_archivo("reporte.txt")
            nombre = input("Ingresa el nombre del archivo de entrada 
            (con extensión .txt): ")
            archivo = f"test/{nombre}"
            if os.path.exists(archivo):
                with open(archivo, "r", encoding="utf-8") as archivo_texto:
                    texto = archivo_texto.read().replace('\n', ' ')                    
                    lexemas = tokenizador.tokenizador(texto)
                    tabla_lexemas = tokenizador.afd.generar_tabla_lexemas()
                    with open(os.path.join("output","tabla_lexemas.txt"),
                     "w", encoding="utf-8") as archivo_salida:
                        archivo_salida.write(f"{'Lexema':<20} 
                        {'Token':<15} {'Patron'}\n")
                        archivo_salida.write("=" * 50 + "\n")
                        for entrada in tabla_lexemas:
                            archivo_salida.write(f"{entrada['Lexema']:<20} 
                            {entrada['Token']:<15} {entrada['Patron']}\n")
                    print("-"*64)
                    resultado_sentimiento = analizar_sentimiento
                    (lexemas, tabla_sentimientos, tokenizador)
                    print("Detección de sentimientos: ")
                    print(f"Sentimiento general: 
                    {resultado_sentimiento['sentimiento']}")
                    print(f"Puntaje total: 
                    {resultado_sentimiento['puntaje_total']}")
                    print(f"Palabras positivas: 
                    {resultado_sentimiento['palabras_positivas']}")
                    print(f"Palabras negativas: 
                    {resultado_sentimiento['palabras_negativas']}")

                    resultado_protocolo = verificar_protocolo
                    (lexemas,tokenizador)
                    print(f"Verificación del protocolo de 
                    Atención (Agente): ")
                    print(f" - Saludo: {'OK' 
                    if resultado_protocolo['saludo'] else 'No detectado'}")
                    print(f" - Identificación: {'OK' if 
                    resultado_protocolo['identificacion'] 
                    else 'No detectado'}")
                    print(f" - Despedida: {'OK' 
                    if resultado_protocolo['despedida'] else 
                    'No detectado'}")
                    print(f" - Palabras rudas: 
                    {'Detectadas' 
                    if resultado_protocolo['palabras_prohibidas'] 
                    else 'Ninguna detectada'}")
                    print("-"*64)

                    output_filename = "reporte.txt"
                    with open(os.path.join("output",output_filename), 
                    "w", encoding="utf-8") as archivo_salida:
                        archivo_salida.write(f"Sentimiento general: 
                            {resultado_sentimiento['sentimiento']}\n")
                        archivo_salida.write(f"Puntaje total: 
                            {resultado_sentimiento['puntaje_total']}\n")
                        archivo_salida.write(f"Palabras positivas: 
                            {resultado_sentimiento['palabras_positivas']}\n")
                        archivo_salida.write(f"Palabras negativas: 
                            {resultado_sentimiento['palabras_negativas']}\n")
                        archivo_salida.write(f"Verificación del 
                            protocolo de Atención (Agente):\n")
                        archivo_salida.write(f" - Saludo: 
                            {'OK' if resultado_protocolo['saludo'] 
                            else 'No detectado'}\n")
                        archivo_salida.write(f" - Identificación: 
                            {'OK' if resultado_protocolo['identificacion'] 
                            else 'No detectado'}\n")
                        archivo_salida.write(f" - Despedida: 
                            {'OK' if resultado_protocolo['despedida'] 
                            else 'No detectado'}\n")
                        archivo_salida.write(f" - Palabras rudas: 
                            {'Detectadas' 
                            if resultado_protocolo['palabras_prohibidas'] 
                            else 'Ninguna detectada'}\n")

                    print(f"[*] Archivo de salida guardado en: 
                                {output_filename}")
                    afd_json = "afd.json"
                    tokenizador.afd.guardar_en_json(afd_json)
                    print(f"[*] AFD guardado en: {afd_json}")
                    print("[*] Tabla de lexemas guardada en:
                             tabla_lexemas.txt")
            else:
                print(f"El archivo {archivo} no existe.")

        elif opcion == '2':
            tokenizador.afd.vaciar() 
            agregar_palabra_sentimiento(tabla_sentimientos, tokenizador.afd)
            tokenizador = TokenizadorAFD()
            tabla_sentimientos = cargar_palabras_y_puntajes
            ("palabras_y_puntajes.txt")

        elif opcion == '3':
            tokenizador.afd.vaciar() 
            eliminar_palabra_sentimiento(tabla_sentimientos)
            tokenizador = TokenizadorAFD()
            tabla_sentimientos = cargar_palabras_y_puntajes
            ("palabras_y_puntajes.txt")

        elif opcion == '4':
            print("Saliendo del programa.")
            break

        else:
            print("Opción no válida. Intente de nuevo.")

if __name__ == "__main__":
    main()
        
\end{verbatim}
\section*{Código Principal del Análisis de Sentimientos y Protocolo de Atención}

Este código implementa el flujo principal de un sistema de análisis de sentimientos y
verificación de protocolo de atención utilizando un AFD (Autómata Finito Determinista) y una
tabla de sentimientos.

\subsection*{Módulos Importados}
\begin{itemize}
	\item \texttt{os}: Proporciona funciones para interactuar con el sistema de archivos.
	\item \texttt{models.afd}: Define el AFD y su tokenizador.
	\item \texttt{models.file\_utils}: Contiene utilidades para agregar y eliminar palabras de la tabla de sentimientos y manejar archivos.
	\item \texttt{models.menu}: Contiene el menú de opciones.
	\item \texttt{models.analisis}: Incluye funciones para analizar sentimientos y verificar protocolos.
\end{itemize}

\subsection*{Función \texttt{main()}}
La función \texttt{main()} controla el flujo del programa a través de un bucle que ofrece las siguientes opciones:

\begin{itemize}
	\item \textbf{Opción 1:} Procesar un archivo de texto de entrada.
	      \begin{itemize}
		      \item Limpia el AFD y carga la tabla de sentimientos
		            desde \texttt{palabras\_y\_puntajes.txt}.

		      \item Elimina archivos temporales como \texttt{tabla\_lexemas.txt}, \texttt{afd.json} y \texttt{reporte.txt}.
		      \item Lee el archivo de texto especificado, tokeniza el contenido usando el AFD, y guarda la tabla de lexemas generada en \texttt{tabla\_lexemas.txt}.
		      \item Realiza el análisis de sentimientos e imprime los resultados en consola.
		      \item Verifica el protocolo de atención y muestra en consola si el agente cumple con los requisitos (saludo, identificación, despedida y la ausencia de palabras prohibidas).
		      \item Guarda el reporte de resultados en \texttt{reporte.txt} y las transiciones usadas del AFD en un archivo JSON.
	      \end{itemize}

	\item \textbf{Opción 2:} Agregar una nueva palabra a la tabla de sentimientos y al AFD.
	      \begin{itemize}
		      \item Limpia el AFD, llama a \texttt{agregar\_palabra\_sentimiento}, recarga el AFD y la tabla de sentimientos.
	      \end{itemize}

	\item \textbf{Opción 3:} Eliminar una palabra de la tabla de sentimientos y el archivo de persistencia.
	      \begin{itemize}
		      \item Limpia el AFD, llama a \texttt{eliminar\_palabra\_sentimiento}, y recarga el AFD y la tabla de sentimientos.
	      \end{itemize}

	\item \textbf{Opción 4:} Salir del programa.
\end{itemize}

La función \texttt{main()} permite interactuar con el sistema de análisis, actualizando el AFD
y la tabla de sentimientos de acuerdo con las modificaciones realizadas.

\subsection*{Notas Importantes}
El código maneja los resultados de análisis de sentimientos y verificación de protocolo tanto
en la consola como en archivos de texto para su revisión.

\subsection{Resultados}
El siguiente es el resultado del análisis de un ejemplo de conversación entre un agente y un
cliente, en el cual se detecta un sentimiento general positivo. Los tokens asociados a palabras
positivas superan en número y en puntaje a las palabras negativas, lo que sugiere una
interacción satisfactoria. Además, se verifica que el agente ha cumplido con el protocolo de
atención al cliente, incluyendo saludo, identificación, y despedida, sin utilizar palabras
prohibidas.

\subsection*{Link del Repositorio:}
\href{https://github.com/AlexArce2000/tokenizador-tp-compiladores}{Repositorio Github}

\subsection*{Caso de ejemplo}
\subsection*{Entrada:}
\begin{verbatim}
-----------------------------------------------------------------    
Agente: ¡Buenos días! Gracias por contactar con el servicio al 
cliente de ConexiónNet. Mi nombre es Johanna. ¿Cómo puedo 
ayudarle hoy?

Cliente: Hola, estoy teniendo algunos problemas con mi 
internet. Pago por 100 Mbps, pero últimamente he notado que 
la velocidad es mucho más baja.

Agente: Lamento escuchar eso, y estoy aquí para ayudarle 
a resolverlo. ¿Podría proporcionarme su nombre completo 
y el número de cuenta para que pueda revisar su situación?

Cliente: Claro, soy Juan Pérez y mi número de cuenta es 
12345678.

Agente: ¡Gracias, Juan! Voy a revisar su cuenta ahora mismo.
Veo que efectivamente tiene un plan de 100 Mbps. 
Déjeme hacer una prueba de línea para verificar cómo está 
funcionando su conexión en este momento.

Cliente: Sí, por favor. Me gustaría que se resolviera pronto.

Agente: ¡Entiendo perfectamente! Estoy realizando la prueba.
Vaya, parece que hay algunas fluctuaciones en la señal. 
No se preocupe, eso es algo que podemos solucionar. Voy a 
reiniciar su línea desde aquí, lo que puede ayudar a estabilizar 
la conexión. Esto tomará solo un minuto, y perderá la conexión 
brevemente. ¿Le parece bien?

Cliente: Sí, eso suena bien. Espero que funcione.

Agente: ¡Genial! Estoy reiniciando su línea. Listo. 
El reinicio se ha completado. Ahora, por favor, reinicie también 
su módem y router manualmente. Simplemente desconéctelos de la 
corriente durante 30 segundos y luego vuelva a conectarlos. Esto 
debería refrescar la señal.

Cliente: Está bien, lo haré. ¿Y si eso no soluciona el problema?

Agente: Si después de reiniciar su equipo el problema persiste, 
no se preocupe. Programaremos una visita de un técnico a su 
domicilio para asegurarnos de que todo esté en orden. Queremos 
que disfrute del servicio por el que está pagando.

Cliente: Eso suena bien. Solo quiero que funcione como debería.

Agente: Lo comprendo completamente, Juan. Estoy segura de que 
podremos resolver esto. Espero que el reinicio funcione, pero 
si no, estaremos aquí para ayudarle. ¿Hay algo más en lo que 
pueda asistirte hoy?

Cliente: No, eso es todo. Solo espero que se solucione pronto.

Agente: ¡Perfecto! Agradezco su paciencia y confianza. No dude 
en llamarnos si necesita algo más. Le deseo un excelente día 
y que su servicio se estabilice pronto. 
¡Estamos aquí para ayudarle!

Cliente: Gracias, adiós.

Agente: ¡Adiós, Juan! Que tenga un buen día.
----------------------------------------------------------------- 
\end{verbatim}

\subsection*{Salida:}
\begin{tcolorbox}[colback=gray!10, colframe=gray!80, sharp corners, boxrule=0.5pt]
	\begin{verbatim}
Sentimiento general: Positivo
Puntaje total: 12
Palabras positivas: ['bien', 'bien', 'genial', 'bien', 'bien', 
'perfecto', 'paciencia', 'excelente']
Palabras negativas: ['problema', 'problema']
Verificación del protocolo de Atención (Agente):
    - Saludo: OK
    - Identificación: OK
    - Despedida: OK
    - Palabras rudas: Ninguna detectada
\end{verbatim}
\end{tcolorbox}

\section{Conclusión}

Este proyecto implementa un sistema integral de análisis de sentimientos y verificación de
protocolos de atención mediante el uso de un Autómata Finito Determinista (AFD) y una tabla de
sentimientos. La arquitectura del sistema permite la identificación y clasificación de palabras
con carga emocional, así como la detección de palabras inapropiadas en las interacciones.
Además, el sistema verifica el cumplimiento de los elementos fundamentales del protocolo de
atención (saludo, identificación y despedida), proporcionando una evaluación detallada de las
respuestas del agente.

La flexibilidad del sistema para agregar o eliminar palabras de la tabla de sentimientos y
actualizar el AFD en tiempo real lo hace adaptable a nuevas necesidades o criterios específicos
de análisis. Esta capacidad de modificación y persistencia de datos asegura que el sistema se
pueda ajustar continuamente para mejorar la precisión de las evaluaciones y adaptarse a
diferentes contextos de uso.

En conjunto, este sistema representa una herramienta robusta y adaptable para el análisis
automatizado de interacciones en centros de contacto, con potencial de ser extendido o
integrado en aplicaciones más amplias de procesamiento de lenguaje natural (PLN) o de mejora de
calidad en la atención al cliente.


\end{document}
