\section{Implementacion}

\subsection{Arquitectura}
% TODO: agregar una descripcion de las clases principales
El sistema está estructurado de la siguiente manera:

\begin{tcolorbox}[colback=gray!10, colframe=gray!80, sharp corners, boxrule=0.5pt]
\begin{verbatim}
+-- audios
|   +-- audio-texto.py
|   +-- prueba-texto1-negativo.m4a
|   +-- prueba-texto2-positivo.m4a
+-- data
|   +-- despedidas.txt
|   +-- identificaciones.txt
|   +-- palabras_prohibidas.txt
|   +-- palabras_y_puntajes.txt
|   +-- saludos.txt
|
+-- doc
|   +-- ensayo.pdf
|
+-- models
|   +-- afd.py
|   +-- analisis.py
|   +-- file_utils.py
|   +-- menu.py
|   +-- sentimientos.py
|   +-- __init__.py
|
+-- output
|   +-- afd.json
|   +-- reporte.txt
|   +-- tabla_lexemas.txt
|
+-- test
|    +-- prueba-texto1-negativo.txt
|    +-- prueba-texto2-positivo.txt
|    +-- prueba-texto3-positivo.txt
|    +-- prueba-texto4-neutral.txt
|    +-- prueba.txt
|
|- tokenizador.py
|- README.md

\end{verbatim}
\end{tcolorbox}

\subsection{Tokenizacion}
\subsubsection{Tokenizador AFD}
\subsubsection{Tokenizador Hashmap}

\subsection{Correccion de tokens desconocidos}

\subsection{Analisis de sentimiento y protocolo de atención}

\subsection{Generacion de informes}

\subsubsection{Tabla de Sentimientos}
Se diseñó una tabla de sentimientos que asigna puntajes específicos a palabras según su
connotación, lo cual permite un análisis más detallado y preciso de las emociones expresadas en
las interacciones. Estos datos, pre-cargados desde un archivo de texto para garantizar su
persistencia, incluyen la flexibilidad de agregar o eliminar palabras según se requiera,
    facilitando así la personalización y actualización continua del análisis de sentimiento.

    \newpage

    \subsection*{Notas Importantes}
    El código maneja los resultados de análisis de sentimientos y verificación de protocolo tanto
    en la consola como en archivos de texto para su revisión.
