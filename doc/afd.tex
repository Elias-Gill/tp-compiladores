El algoritmo implementado es un sistema de análisis de interacciones en contact centers que
utiliza técnicas de tokenización y análisis de sentimientos. A continuación se detallan los
componentes y el funcionamiento del algoritmo:

\subsubsection{Autómata Finito Determinista (AFD)}
El AFD es la estructura central que permite la clasificación de las palabras en diferentes
categorías como saludos, identificaciones, despedidas, palabras positivas, negativas y
prohibidas. Se inicia con un estado inicial (q0) y tiene varios estados finales que
corresponden a las categorías mencionadas. Las transiciones se definen para cada palabra de
interés, lo que permite al AFD aceptar o rechazar lexemas basándose en las palabras que se
encuentran en el texto analizado.

\begin{figure}[h]
	\centering
	\begin{tikzpicture}[
			shorten >=1pt,
			node distance=2.5cm,
			on grid,
			auto
		]
		\node[state, initial] (q0) {start};
		\node[state] (q1) [right=of q0] {$q_1$};
		\node[state] (q2) [right=of q1] {$q_2$};
		\node[state] (q3) [right=of q2] {$q_3$};
		\node[state, accepting] (q4) [right=of q3] {$q_4$};

		\path[->]
		(q0) edge node {h} (q1)
		(q1) edge node {o} (q2)
		(q2) edge node {l} (q3)
		(q3) edge node {a} (q4);
	\end{tikzpicture}
	\caption{Transiciones para la frase "hola" tipo SALUDO}
\end{figure}


El \textbf{AFD} (Autómata Finito Determinista) implementado en este proyecto tiene como
objetivo reconocer frases o palabras clave dentro de un texto, asignando un \textit{token} con
tipo semántico (saludo, despedida, etc.) y puntuación según una tabla predefinida. Este
autómata no está construido manualmente, sino que se genera dinámicamente a partir de listas de
palabras/frases, agrupadas por tipo.

\subsubsection{Construcción del AFD}

Para cada palabra o frase $\omega = c_1 c_2 \ldots c_n$ asociada a un tipo $T$:
\begin{enumerate}
	\item Se comienza desde el estado inicial $q_0$.
	\item Por cada caracter $c_i$ se crea un estado nuevo (si no existe) y una transición desde el estado actual.
	\item El último estado creado $q_n$ se marca como \textit{final}, y se le asigna el tipo de token $T$ y el puntaje de la frase.
\end{enumerate}

Así, frases distintas que comparten prefijos (por ejemplo, \texttt{"buen"} y \texttt{"buenos
	días"}) comparten transiciones, optimizando el tamaño del autómata.

Una vez construido, el AFD se guarda como un archivo \texttt{afd.json}, permitiendo depuración
y visualización externa.

\subsubsection{Reconocimiento de tokens}

Para tokenizar un texto:
\begin{enumerate}
	\item Se itera carácter por carácter desde la posición actual $i$ del texto.
	\item Se mantienen múltiples rutas activas en paralelo (cada una con estado actual, acumulador de caracteres y longitud).
	\item Cuando una ruta alcanza un estado final, se guarda como el mejor token candidato hasta el momento.
	\item Si se encuentra un token más largo, se reemplaza el candidato anterior.
	\item Al terminar el recorrido, si se detectó un token válido, se retorna. Si no, se interpreta como una palabra genérica.
\end{enumerate}

\subsubsection{Características adicionales}

\begin{itemize}
	\item El autómata diferencia hablantes mediante tokens como \texttt{agente:} o \texttt{cliente:}.
	\item Se identifican signos de puntuación como tokens individuales.
	\item Si una palabra no es reconocida por el AFD, se busca en la tabla de sentimientos como palabra suelta.
\end{itemize}

\subsection{Estructura de Datos}
El AFD tiene un diccionario para almacenar las transiciones entre estados y otro para los
identificadores de lexemas. Un contador permite llevar el registro de cuántas veces se han
reconocido las palabras de cada categoría. La tabla de sentimientos utiliza una lista enlazada
donde es posible agregar o eliminar palabras y ponderaciones mediante el menú del sistema.

\subsubsection{Definición de Estados y Transiciones}

Se definieron múltiples estados finales en el AFD, cada uno asociado a distintas categorías de
palabras: saludos, identificaciones, despedidas, palabras prohibidas, palabras positivas y
palabras negativas. Esta organización estructurada permite un análisis claro y eficiente de las
interacciones con los clientes. Las palabras correspondientes a cada categoría se cargan
previamente desde archivos de texto, listos para su uso en el análisis.

Las transiciones del autómata se implementaron para reconocer un conjunto específico de
palabras, seleccionadas cuidadosamente para representar las categorías relevantes dentro del
contexto de atención al cliente y la verificación del protocolo que debe seguir el agente en
diversas situaciones.

