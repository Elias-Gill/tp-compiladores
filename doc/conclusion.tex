\section{Conclusión}

Este proyecto implementa un sistema integral de análisis de sentimientos y verificación de
protocolos de atención mediante el uso de un Autómata Finito Determinista (AFD) y una tabla de
sentimientos. La arquitectura del sistema permite la identificación y clasificación de palabras
con carga emocional, así como la detección de palabras inapropiadas en las interacciones.
Además, el sistema verifica el cumplimiento de los elementos fundamentales del protocolo de
atención (saludo, identificación y despedida), proporcionando una evaluación detallada de las
respuestas del agente.

La flexibilidad del sistema para agregar o eliminar palabras de la tabla de sentimientos y
actualizar el AFD en tiempo real lo hace adaptable a nuevas necesidades o criterios específicos
de análisis. Esta capacidad de modificación y persistencia de datos asegura que el sistema se
pueda ajustar continuamente para mejorar la precisión de las evaluaciones y adaptarse a
diferentes contextos de uso.

En conjunto, este sistema representa una herramienta robusta y adaptable para el análisis
automatizado de interacciones en centros de contacto, con potencial de ser extendido o
integrado en aplicaciones más amplias de procesamiento de lenguaje natural (PLN) o de mejora de
calidad en la atención al cliente.
