\section{Conclusión} 
Este proyecto implementa un sistema para el análisis de sentimientos y la verificación del
cumplimiento de protocolos en conversaciones, utilizando una tabla de sentimientos y un
autómata finito determinista (AFD). El sistema permite detectar saludos, identificaciones,
despedidas, frases con carga emocional y palabras prohibidas, evaluando si el agente cumple con
los criterios del protocolo establecido.

Se compararon dos enfoques para la tokenización del texto: uno basado en AFD y otra solucion
híbrida principalmente enfocada en búsqueda mediante tablas hash. Ambos cumplen con los
objetivos y presentan ventajas propias, pero se optó por el tokenizador basado en AFD como
implementación principal, debido a su mayor precisión al reconocer frases largas y resolver
ambigüedades.

El sistema es fácilmente adaptable, ya que permite agregar o eliminar palabras y actualizar la
tabla de símbolos, lo que facilita su ajuste a distintos contextos o criterios de análisis.
