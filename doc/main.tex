\documentclass[12pt,a4paper]{article}

% \usepackage[spanish, es-tabla]{babel}

% Fuentes y estilos
\usepackage[utf8]{inputenc}
\usepackage[version=3]{mhchem}
\usepackage[journal=jacs]{chemstyle}
\usepackage{amsmath}
\usepackage{amsfonts}
\usepackage{amssymb}
\usepackage{makeidx}
\usepackage[left=3cm,right=3cm,top=3cm,bottom=3cm]{geometry}

%Formato del títulos
\usepackage{titlesec}
\usepackage{enumitem}
\titleformat*{\section}{\bfseries\large}
\titleformat*{\subsection}{\bfseries\normalsize}

%%%% Entorno código - No usado de momento
\usepackage{tcolorbox}
\usepackage{textcomp}
\usepackage{listings}
\usepackage{xcolor}
\usepackage{verbatim}
\usepackage{listings}
\usepackage{xcolor}  % Opcional, para colores
\lstset{
    language=Python,
    basicstyle=\ttfamily\small,
    keywordstyle=\color{blue},
    stringstyle=\color{red},
    commentstyle=\color{gray},
    showstringspaces=false,
    breaklines=true
    extendedchars=true,
    inputencoding=utf8,
    literate=
    {á}{{\'a}}1
    {é}{{\'e}}1
    {í}{{\'i}}1
    {ó}{{\'o}}1
    {ú}{{\'u}}1
    {ñ}{{\~n}}1
    {ü}{{\"u}}1,
}

%%% Grafos
\usepackage{tikz}
\usetikzlibrary{automata, positioning, arrows}

%% Algoritmos 
\usepackage{algorithm}
\usepackage{algorithmic} 

% Imágenes
\usepackage{graphicx}
\usepackage{lmodern}

% Pies de pagina y referencias
\usepackage[stable]{footmisc}
\usepackage[section]{placeins}
\usepackage{fancyhdr}
\usepackage[colorlinks=true, linkcolor=black, urlcolor=blue, citecolor=black]{hyperref}

\begin{document}
%------------------------ Fin del preambulo -------------------------------

%%%%%%%%%%%%%%%%%%%%%%%%%%%%%%%%%%%%
%           CARATULA               %
%%%%%%%%%%%%%%%%%%%%%%%%%%%%%%%%%%%%
\begin{titlepage}
	\begin{center}

	% Logo institucional (opcional)
	% \includegraphics[width=0.4\textwidth]{logo_universidad.png}
	% \vspace*{1cm}

	% Título principal
	\vspace*{2cm} % Más espacio arriba del título
	{\LARGE \textbf{Análisis de Interacciones en Contact Centers} \\
		\vspace{0.3cm} % Espacio entre líneas del título
		\large con Speech Analytics y Tokenización \\}
	\vspace{1.5cm} % Espacio después del título

	% Autores (formato de dos columnas)
	\begin{minipage}{0.8\textwidth}
		\centering
		\large
		\textbf{Autores:} \\
		\vspace{0.5cm} % Espacio entre autores
		Elías Sebastián Gill Quintana \\
		\vspace{0.2cm} % Espacio entre autores
		% \textbf{Coautor:} \\
		Pablo Caballero [continar] \\ % Completa con el apellido
	\end{minipage}
	\vfill

	% Información académica
	\large
	\textbf{Trabajo Práctico} \\
	\vspace{0.3cm}
	\textbf{Asignatura:} Diseño de Compiladores \\
	\vspace{0.3cm}
	\textbf{Profesor:} Sergio Andrés Aranda Zemán \\
	\vspace{1cm}

	% Fecha
	\textbf{Mayo, 2025}
	\vspace*{1cm} % Espacio final

\end{center}

\end{titlepage}
\newpage

%%%%%%%%%%%%%%%%%%%%%%%%%%%%%%%%%%%%
%             Indice               %
%%%%%%%%%%%%%%%%%%%%%%%%%%%%%%%%%%%%

\tableofcontents
\newpage

%%%%%%%%%%%%%%%%%%%%%%%%%%%%%%%%%%%%
%      Inicio del documento        %
%%%%%%%%%%%%%%%%%%%%%%%%%%%%%%%%%%%%

% --- Secciones del documento --- 

\section{Introducción}
Este trabajo presenta un sistema modular para el análisis de interacciones en contact centers,
combinando técnicas de Speech Analytics y tokenización. El sistema procesa transcripciones de
audio para clasificar palabras en categorías semánticas como saludos, despedidas,
identificaciones y términos con carga emocional, además de realizar análisis de sentimientos.

La implementación base utiliza un Autómata Finito Determinista (AFD) para reconocer patrones
conversacionales complejos y verificar protocolos de atención, con capacidad de extensión
mediante nuevos estados. Como contribución adicional, se desarrolló una versión alternativa
basada en tablas hash (técnicamente es una implementación híbrida dado su uso de expresiones
regulares para extraer palabras de signos de puntuación), optimizada para búsquedas léxicas en
tiempo constante y compatible con la misma interfaz del tokenizador basado en AFD. Esta
arquitectura dual permite comparar ambos enfoques en términos de eficiencia y precisión,
seleccionando la implementación deseada en tiempo de ejecución sin modificar el sistema. El
diseño no solo enriquece el valor académico del proyecto, sino que también proporciona
evidencia empírica sobre las ventajas de cada enfoque en distintos escenarios operativos.
 % y resumen
\section{Arquitectura}

El sistema cuenta con los siguientes módulos:

\begin{itemize}
	\item Tabla de sentimientos
	\item Tokenizador
	\item Analizador de sentimientos
	\item Reportes (menú principal)
\end{itemize}

Cada modulo esta diseñado para realizar funciones especificas de manera modular, por lo que
se pueden realizar distintas implementaciones siempre y cuando se sigan las interfaces comunes.

\subsection{Flujo del programa}
El programa inicia cargando la tabla con las palabras necesarias para realizar el análisis de
sentimientos. A continuación, se selecciona e instancia el tipo de tokenizador, según los
argumentos proporcionados al momento de ejecutar el programa.

Si se indica un archivo como argumento, su contenido es procesado automáticamente. En caso
contrario, el programa entra en modo interactivo, permitiendo al usuario ingresar texto
directamente por consola.

La entrada es primero tokenizada utilizando el tokenizador seleccionado. Con la lista completa
de tokens disponibles, aquellos que son clasificados como \texttt{desconocidos} son presentados
al usuario para su corrección o descarte. Para cada token desconocido, el usuario puede elegir
entre las siguientes opciones:

\begin{itemize}
	\item Ignorar el token.
	\item Añadirlo a la tabla de sentimientos.
	\item Corregirlo utilizando una de las palabras sugeridas (si existen).
\end{itemize}

En caso de optar por la corrección mediante sugerencias, se muestra una lista de palabras
similares, calculadas utilizando la distancia de Levenshtein.

Finalmente, se realiza el análisis de sentimiento utilizando los tokens generados. El resultado
del análisis se muestra por consola. Se guardan en archivos de texto plano tanto un resumen del
procesamiento como la lista de tokens resultantes del proceso de tokenizacion.

\subsection{Tabla de Sentimientos}
La tabla de sentimientos es la estructura que contiene las palabras y frases categorizadas por
tipo junto con los puntajes asociados a cada una. Estos puntajes son utilizados posteriormente
para calcular la polaridad del texto analizado.

Además de su rol como base de conocimiento, esta tabla incluye las siguientes funcionalidades
de conveniencia que facilitan su uso y actualización dinámica durante el análisis:

\begin{itemize}
	\item \textbf{Búsqueda de palabras o frases:} permite consultar rápidamente si una palabra
	      o frase está presente en la tabla y recuperar su puntaje asociado.

	\item \textbf{Cálculo de correcciones recomendadas:} Calcula sugerencias de posibles
	      correcciones utilizando la distancia de Levenshtein, proponiendo palabras similares ya
	      registradas.

	\item \textbf{Incorporación de nuevas entradas:} el usuario puede añadir palabras con
	      su respectivo puntaje de sentimiento, ampliando así progresivamente la tabla a medida
	      que se utiliza el programa.
\end{itemize}

\subsection{Tokenizador}
El tokenizador es el componente encargado de segmentar el texto de entrada en unidades léxicas
denominadas \emph{tokens}. Es capaz de identificar y clasificar distintos tipos de tokens,
tales como palabras simples, frases compuestas, signos de puntuación, números o símbolos
especiales. Además, posee mecanismos para resolver ambigüedades en aquellos casos donde
múltiples tipos pudieran coincidir con un mismo patrón textual.

El tokenizador recibe una entrada en forma de texto plano y retorna una lista estructurada de
tokens.

\subsubsection{Tokens}
Cada token incluye la siguiente información:

\begin{itemize}
	\item \textbf{Texto del token:} la representación literal extraída del texto original.
	\item \textbf{Tipo:} categoría asignada al token (por ejemplo, palabra conocida, palabra desconocida, puntuación, etc.).
	\item \textbf{Puntaje:} valor numérico asociado al token, obtenido de la tabla de sentimientos si corresponde.
\end{itemize}

\subsection{Analisis de sentimiento}
Una vez obtenida la lista completa de tokens, se procede al análisis de estos para determinar
su validez y relevancia en el contexto del análisis de sentimientos.

Si se detectan tokens de tipo \emph{desconocido} (que no se encuentran registrados en la tabla
de sentimientos), el sistema solicita al usuario elegir si ignorarlos o proceder a su
corrección de manera manual. En caso de que el usuario decida corregir los tokens desconocidos,
este módulo se encarga de gestionar todo el proceso de corrección.

El análisis de sentimiento se realiza de forma diferenciada para cada hablante, lo que permite
evaluar en detalle el desempeño individual del cliente y del agente dentro de la conversación.

\subsection{Reportes}
El módulo de reportes se encarga de generar archivos con los resultados del proceso de
tokenización y del análisis de sentimiento. Además, presenta un resumen conciso directamente en
la terminal.

%%%%%%%%%%%%%%%%%%%%%%%%%%%%%%%%%%%%%%%%%%%%%%%%%
% Decisiones clave en la parte de implemetacion %
%%%%%%%%%%%%%%%%%%%%%%%%%%%%%%%%%%%%%%%%%%%%%%%%%
\subsection{Decisiones clave}

\subsubsection{Tipos de tokens}
Los tipos de tokens identificados por el tokenizador son:

\begin{enumerate}
	\item {\footnotesize\textbf{TOKEN\_SALUDO}}: frase o palabra que corresponde a un saludo.
	\item {\footnotesize\textbf{TOKEN\_DESPEDIDA}}: frase o palabra de despedida.
	\item {\footnotesize\textbf{TOKEN\_IDENTIFICACION}}: indica la identificación de alguna de las partes.
	\item {\footnotesize\textbf{TOKEN\_PROHIBIDA}}: palabra o frase grosera.
	\item {\footnotesize\textbf{TOKEN\_SENTIMIENTO}}: palabra individual con un sentimiento asociado.
	\item {\footnotesize\textbf{TOKEN\_DESCONOCIDO}}: palabra no registrada en la tabla de sentimientos.
	\item {\footnotesize\textbf{TOKEN\_AGENTE}}: indica cambio de hablante a un agente.
	\item {\footnotesize\textbf{TOKEN\_CLIENTE}}: indica cambio de hablante a un cliente.
	\item {\footnotesize\textbf{TOKEN\_SIGNO\_PUNTUACION}}: representa signos de puntuación.
\end{enumerate}

\subsubsection{Detección de frases}
Como parte del diseño del sistema, se asumió el desafío de identificar no solo palabras
aisladas, sino también frases cortas como \textit{"buenos días"}, con el fin de facilitar la
identificación de protocolos como saludos o identificaciones.

Esto implica que el sistema debe resolver ambigüedades, por ejemplo, entre \textit{"buenos"}
(sentimiento) y \textit{"buenos días"} (saludo).

Esto influye directamente en la implementación del mecanismo de tokenizacion, dado que el
tokenizador debe de priorizar correctamente las frases completas por sobre palabras sueltas.

\subsubsection{Mecanismo de tokenizacion}
Dado el tiempo disponible, se desarrollaron dos tipos de tokenizadores: uno basado en autómatas
finitos deterministas (AFD) y otro basado en tablas hash.

Ambos implementan una interfaz común con el método \textbf{\textit{tokenizar()}}, el cual
devuelve una lista con los tokens analizados.

Se consideró la posibilidad de utilizar un método \textbf{\textit{next\_token()}}, pero se optó
por una solución más simple que retorne todos los tokens de una vez.

Gracias a su diseño modular, los tokenizadores son fácilmente intercambiables, lo que permite
realizar comparaciones sobre facilidad de mantenimiento, extensibilidad, implementación y
rendimiento.

\subsubsection{API de la tabla de sentimientos}
La tabla de sentimientos almacena los distintos tipos de frases (saludos, sentimientos, etc.)
en diccionarios separados. Además, cuenta con funciones auxiliares, entre ellas la más
importante es \textbf{\textit{buscar\_palabra()}}, utilizada especialmente por el tokenizador
basado en hashmaps.

Esta función garantiza el siguiente orden de prioridad al buscar una palabra:

\begin{enumerate}
	\item \footnotesize{TOKEN\_SALUDO}
	\item \footnotesize{TOKEN\_DESPEDIDA}
	\item \footnotesize{TOKEN\_IDENTIFICACION}
	\item \footnotesize{TOKEN\_PROHIBIDA}
	\item \footnotesize{TOKEN\_SENTIMIENTO}
	\item \footnotesize{TOKEN\_DESCONOCIDO}
\end{enumerate}

Por lo tanto, si una palabra aparece en múltiples categorías, se prioriza el tipo más
específico.

\subsubsection{Solución de ambigüedades}
Para identificar frases en el tokenizador basado en hashmaps, se emplea un buffer de hasta 3
palabras, lo que permite dar prioridad a aquellas que sean mas largas. Si no se encuentra una
coincidencia con 3 palabras, se intenta con 2 y luego con 1.

En el caso del tokenizador AFD se utiliza el algoritmo de \textit{longest match}, es decir, se
continúa la lectura mientras el autómata permita avanzar hacia un estado válido, favoreciendo
la identificación de frases completas. Más detalles sobre el algoritmo de "longest match" y el
proceso de carga del AFD se explican en las secciones siguientes.

La resolucion de palabras o frases repetidas se realiza
mediante la carga del AFD en el siguiente orden:

\begin{verbatim}
self._cargar_frases_al_afd(self.tabla.palabras, TIPO_SENTIMIENTO)
self._cargar_frases_al_afd(self.tabla.palabras_prohibidas, TIPO_PROHIBIDA)
self._cargar_frases_al_afd(self.tabla.identificaciones, TIPO_IDENTIFICACION)
self._cargar_frases_al_afd(self.tabla.despedidas, TIPO_DESPEDIDA)
self._cargar_frases_al_afd(self.tabla.saludos, TIPO_SALUDO)
\end{verbatim}

Así, si una palabra aparece en más de un conjunto, se considera como "canónica" aquella
asociada al tipo más específico.

\section{Implementación}
\noindent\textbf{Nota:} Esta sección solo se describe la implementación y funcionamiento de los
módulos principales. El código fuente de dichas clases se proporciona en la
\hyperref[sec:Anexos]{sección de Anexos} (página \pageref{sec:Anexos}).

\subsection{Converión de audio a texto}

Para realizar la transcripcion de audio a texto se utilizo el dictado por vos del teclado de
Google. Luego, con la finalidad de identificar los hablantes, se proporciono el
texto transcripto a ChatGPT utilizando el siguiente prompt:

\begin{verbatim}
Identifica y separa los diálogos entre el Agente y 
el Cliente en el siguiente texto:
<insertar el texto>
Dame el resultado en el formato 'Agente:' y 'Cliente:'
\end{verbatim}

\subsection{Estructura del codigo}
El código está estructurado de la siguiente manera:

\begin{tcolorbox}[colback=gray!10, colframe=gray!80, sharp corners, boxrule=0.5pt]
	\begin{verbatim}
|-- main.py
|-- output
|   `-- reportes ...
|-- test
|   `-- textos de prueba ...
`-- tokenizer
    |-- AFDTokenizer.py
    |-- analisis.py
    |-- HashTokenizer.py
    |-- sentiment_symbols
    |   |-- despedidas.txt
    |   |-- identificaciones.txt
    |   |-- palabras_prohibidas.txt
    |   |-- palabras_y_puntajes.txt
    |   `-- saludos.txt
    |-- TablaSentimientos.py
    `-- tokens.py
\end{verbatim}
\end{tcolorbox}

%%%%%%%%%%%%%%%%%%%%%%%%%%%%%%%%%%%
%    La tabla de sentimientos     %
%%%%%%%%%%%%%%%%%%%%%%%%%%%%%%%%%%%
\subsubsection{Tabla de Sentimientos}
\texttt{TablaSentimientos} es la clase encargada de almacenar, gestionar y consultar
diferentes tipos de palabras utilizadas durante el análisis de sentimiento. Para ello, carga en
memoria cinco diccionarios distintos desde archivos persistentes: palabras con puntaje,
saludos, despedidas, identificaciones y palabras prohibidas, a través del método \texttt{\_cargar\_datos()}.

Permite agregar o eliminar palabras con su puntaje asociado mediante los métodos
\texttt{agregar\_palabra(palabra: str, puntuacion: int)} y \texttt{eliminar\_palabra(palabra: str)}.

Provee un método de búsqueda, \texttt{buscar\_palabra(palabra: str)}, el cual retorna el tipo
de palabra (saludo, prohibida, desconocida, etc) y su puntuación de sentimiento.

La clase también ofrece un sistema de sugerencias basado en la similitud de cadenas
(\texttt{difflib.get\_close\_matches}), accesible mediante el método
\texttt{sugerir\_similares()}, facilitando así la corrección y actualización interactiva de la
tabla de sentimientos.

%%%%%%%%%%%%%%%%%%%%%%%%%%%%%%%%%
%      Los tokenizadores        %
%%%%%%%%%%%%%%%%%%%%%%%%%%%%%%%%%
\subsection{Tokenizacion}

%%%%%  Tokenizador AFD  %%%%%
\subsubsection{Tokenizador AFD}

La clase \texttt{AFDTokenizer} transforma un texto de entrada en una secuencia de tokens
semánticamente significativos. Para ello, utiliza un autómata finito determinista (AFD)
generado a partir de una instancia de \texttt{TablaSentimientos}, y aplica un algoritmo de
\textit{longest match} para asegurar coincidencias máximas.

\subsubsection{Construcción del AFD}

El AFD se inicializa mediante el método \texttt{\_build\_afd\_completo()}, que crea un estado
inicial denominado \texttt{start} y lo extiende agregando transiciones según los distintos
tipos de frases registrados en \texttt{TablaSentimientos}. Estas categorías incluyen:

\begin{itemize}
	\item Palabras con puntaje (\texttt{self.tabla.palabras})
	\item Saludos
	\item Despedidas
	\item Identificaciones
	\item Palabras prohibidas
\end{itemize}

La incorporación se realiza con \texttt{\_cargar\_frases\_al\_afd()}, que descompone cada frase
carácter por carácter. Se generan estados intermedios hasta llegar al último carácter, cuyo
estado asociado se marca como final. En dicho estado final se almacenan el tipo de token y, si
corresponde, el puntaje asociado.

\paragraph{\footnotesize{Representación del Estado de Error:}}
El estado de \textit{error}, el cual captura palabras ajenas al vocabulario del AFD, se
implementa implícitamente mediante transiciones no definidas. Cuando un símbolo de entrada no
tiene transición asociada desde el estado actual, el sistema interpreta automáticamente una
transición al estado de error. Esta estrategia optimiza memoria (al evitar representar
transiciones redundantes) y simplifica la lógica de implementación. Esto equivale a modelar
$q_{\text{error}}$ como sumidero universal no almacenado explícitamente.

\subsubsection{Persistencia del AFD}
El método \texttt{\_persistir\_afd()} guarda la estructura resultante en formato JSON bajo la
carpeta \texttt{output}, facilitando su inspección. Cabe denotar que, a fin de simplificar la
implementación, dicha representación no es utilizada para reconstruir el AFD.

\subsubsection{Preprocesamiento del texto}
Antes de tokenizar, se aplica \texttt{\_preprocesar\_hablantes()} para normalizar marcas como
\texttt{agente:} o \texttt{cliente:}, asegurando que sean detectadas como unidades separadas.
Luego, \texttt{\_procesar\_hablante()} se encarga de convertir estas marcas en tokens
especializados.

\subsubsection{Tokenización con AFD y \textit{longest match}}
El núcleo del análisis léxico ocurre en \texttt{tokenizar()}, que recorre el texto carácter por
carácter. Cuando se identifica una palabra candidata, se invoca
\texttt{\_tokenizar\_con\_afd()}.

Este método implementa un algoritmo de \textbf{búsqueda por mayor coincidencia}
(\textit{longest match}) sobre el AFD. A partir de la posición actual, se simulan todas las
transiciones posibles en paralelo:

\begin{itemize}
	\item Cada vez que se alcanza un estado final, se guarda un token candidato.
	\item Se prioriza el token que consuma la mayor cantidad de caracteres consecutivos desde la posición inicial.
	\item Si hay múltiples candidatos válidos, se conserva sólo el más largo.
\end{itemize}

Esto permite resolver ambigüedades como \textit{``buen''}, \textit{``buenos''} y \textit{``buen
	día''}.

\subsubsection{Fallback: \texttt{\_tokenizar\_palabra\_simple()}}
Si \texttt{\_tokenizar\_con\_afd()} no logra reconocer ningún token válido, se recurre al
método \texttt{\_tokenizar\_palabra\_simple()}. Este método agrupa letras consecutivas para
formar una palabra y consulta \texttt{TablaSentimientos.buscar\_palabra()}. Si la palabra no
está registrada, se clasifica como \texttt{DESCONOCIDO}.

\subsubsection{Manejo de signos de puntuación}
Cualquier carácter que no sea alfanumérico ni parte de una palabra compuesta se clasifica como
signo de puntuación mediante \texttt{\_es\_signo\_puntuacion()}, y se encapsula en un token con
tipo \texttt{TOKEN\_SIGNO\_PUNTUACION}.

\subsubsection{Resumen del flujo de análisis léxico}
\begin{enumerate}
	\item Se preprocesa el texto para separar marcas de hablante.
	\item Se recorre el texto carácter a carácter.
	\item Se tokenizan signos de puntuación y hablantes de forma directa.
	\item Se intenta reconocer frases usando el AFD y \texttt{\_tokenizar\_con\_afd()} con \textit{longest match}.
	\item Si el AFD falla, se analiza como palabra suelta con \texttt{\_tokenizar\_palabra\_simple()}.
\end{enumerate}

\subsubsection{Ventajas y limitaciones}
Entre sus principales ventajas encontramos que hace que el proceso de tokenización sea
determinista y predecible. Además, una vez construido, el AFD es especialmente eficiente en
entornos con grandes volúmenes de texto.

La principal desventaja radica en su complejidad, ya que la construcción del AFD puede implicar
un procesamiento costoso tanto en tiempo como en memoria. Además, mantener el AFD en memoria
durante la ejecución puede requerir una cantidad significativa de espacio, especialmente cuando
el conjunto de patrones es grande o altamente variable. Asimismo, la implementación resulta más
compleja de extender y mantener, particularmente cuando se ajustar los criterios de
tokenización.

%%%%%  Tokenizador hashmap  %%%%%
\subsection{Tokenizador Hashmap}
El tokenizador \texttt{HashTokenizer} ofrece una alternativa más simple y directa al uso de un
AFD. Su funcionamiento se basa en una exploración secuencial del texto con búsqueda de
coincidencias en un diccionario (la instancia de \texttt{TablaSentimientos}). Aunque menos
sofisticado que el AFD, su implementación es rápida y efectiva.

\subsubsection{Expresiones regulares}
Se emplean dos expresiones regulares precompiladas (lo que técnicamente convierte la
implementación en híbrida) para agilizar el preprocesamiento del texto:

\begin{itemize}
	\item \texttt{\_hablante\_re}: Detecta marcas de hablante como \texttt{agente:} o
	      \texttt{cliente:} para separarlas del resto del texto.

	\item \texttt{\_palabras\_re}: Segmenta el texto en unidades léxicas válidas. Detecta
	      palabras alfanuméricas y signos de puntuación como unidades separadas.
\end{itemize}

\subsubsection{Algoritmo de tokenización}
El método \texttt{tokenizar()} transforma el texto en una lista de tokens mediante los
siguientes pasos:

\begin{enumerate}
	\item Se normalizan las marcas de hablante agregando espacios alrededor de \texttt{agente:}
	      y \texttt{cliente:}.

	\item Se aplica la expresión regular para obtener una lista de palabras y signos.

	\item Se recorre la lista desde la izquierda aplicando una búsqueda decreciente de frases
	      de longitud 3, 2 y 1 palabras.
\end{enumerate}

\subsubsection{Manejo de marcas de hablante}
Cuando el patrón detecta secuencias como \texttt{agente :} o \texttt{cliente :}, se agrupan y
transforman en un token específico (\texttt{TOKEN\_AGENTE} o \texttt{TOKEN\_CLIENTE}). Este
paso se prioriza antes de intentar reconocer frases.

\subsubsection{Reconocimiento de frases por ventana deslizante}
En cada posición, se intenta formar una frase de longitud 3, luego 2, y finalmente 1 palabra.
Para cada combinación posible:

\begin{itemize}
	\item Se forma la frase concatenando las palabras con espacios.

	\item Se consulta \texttt{tabla.buscar\_palabra(frase)}.

	\item Si se encuentra una coincidencia, se genera un token con el tipo correspondiente y se
	      avanza el cursor \texttt{i} en la longitud de la frase.
\end{itemize}

Este mecanismo puede considerarse una versión simplificada del algoritmo de longest match, sin
necesidad de utilizar una estructura AFD. Sin embargo, también puede presentar errores al
detectar frases que contienen signos de puntuación.

\subsubsection{Fallback: palabra o signo aislado}
Si no se reconoce ninguna frase:

\begin{itemize}
	\item Si el fragmento es un carácter no alfanumérico (\texttt{\textbackslash W}), se
	      clasifica como \texttt{TOKEN\_SIGNO\_PUNTUACION}.

	\item Si es una palabra, se vuelve a consultar en la tabla de sentimientos como palabra
	      individual. Si no se encuentra, se clasifica como \texttt{DESCONOCIDO}.
\end{itemize}

\subsubsection{Ventajas y limitaciones}
Una de las principales ventajas de este enfoque es que su implementación es más directa,
comprensible y extensible. Además, presenta una mayor velocidad de instanciado, ya que no
requiere la construcción previa del AFD. También resulta razonablemente eficaz cuando las
frases significativas tienen una longitud acotada.

Entre sus limitaciones, se encuentra el hecho de que no garantiza coincidencias cuando las
frases contienen signos de puntuación y puede ser menos eficiente (en tiempo de CPU) cuando se
trabaja con grandes volúmenes de texto.

%%%%%%%%%%%%%%%%%%%%%%%%%%%%%%%%%%%%%%%%%
%      Analizador de sentimiento        %
%%%%%%%%%%%%%%%%%%%%%%%%%%%%%%%%%%%%%%%%%
\subsection{Análisis de sentimiento y protocolo de atención}
El análisis de sentimiento se basa en recorrer la secuencia de tokens de la conversación,
asignando puntajes y detectando eventos clave para cliente y agente.

Se distingue quién habla en cada momento y se acumulan las métricas asociadas. También se
identifican eventos importantes para evaluar el protocolo de atención.

\subsubsection{Protocolos y etiquetas especiales}
Se detectan eventos que afectan la evaluación del protocolo:

\begin{itemize}
	\item \textbf{Saludo}: apertura cordial.
	\item \textbf{Despedida}: cierre adecuado.
	\item \textbf{Identificación}: presentación del agente.
	\item \textbf{Palabras prohibidas}: lenguaje inapropiado.
\end{itemize}

Esto se realiza de forma independiente tanto para el agente como para el cliente.

\subsubsection{Estructura del procesamiento}
La función principal recorre los tokens con tres objetivos. Primero, identificar el hablante
actual (cliente o agente). Segundo, acumular puntajes y marcar eventos por hablante. Y tercero,
detectar palabras desconocidas para su posterior manejo.

\begin{verbatim}
def analizar_sentimiento(tokens, tabla_sentimientos):
    resultado = ResultadoConversacion(...)
    palabras_desconocidas = []
    hablante_actual = "agente"

    for token in tokens:
        if token.type == "TOKEN_CLIENTE":
            hablante_actual = "cliente"
            continue
        elif token.type == "TOKEN_AGENTE":
            hablante_actual = "agente"
            continue

        participante = getattr(resultado, hablante_actual)

        if token.type == TOKEN_PROHIBIDA:
            participante.hay_prohibidas = True
        elif token.type == TOKEN_SALUDO:
            participante.hay_saludo = True
        elif token.type == TOKEN_DESPEDIDA:
            participante.hay_despedida = True
        elif token.type == TOKEN_IDENTIFICACION:
            participante.hay_identificacion = True
        elif token.type == TOKEN_DESCONOCIDO:
            palabras_desconocidas.append((hablante_actual, token.valor))

        participante.puntaje_total += token.puntuacion
        resultado.puntaje_total += token.puntuacion
\end{verbatim}

\subsubsection{Corrección de tokens desconocidos}
Durante el análisis, las palabras no encontradas en la tabla de sentimientos se clasifican como
desconocidas. Para cada una se ofrece un menú interactivo con opciones:

\begin{itemize}
	\item \textbf{Agregar manualmente}: ingresar puntaje para agregar la palabra.
	\item \textbf{Corregir con sugerencia}: elegir palabra similar existente.
	\item \textbf{Ignorar}: no modificar el puntaje.
\end{itemize}

\begin{verbatim}
[a] Agregar palabra manualmente")
[c] Corregir usando una sugerencia")
[i] Ignorar palabra")
    Seleccione una opción:
\end{verbatim}

Si el usuario decide agregar o corregir, el puntaje se suma al hablante y a la conversación. Si
ignora, se registra la palabra para análisis futuro.

Este procedimiento permite mejorar y ajustar dinámicamente la tabla de sentimientos para
futuros análisis.

%%%%%%%%%%%%%%%%%%%%%%%%%%%%%%
%      Menu principal        %
%%%%%%%%%%%%%%%%%%%%%%%%%%%%%%
\subsection{\textit{main.py} (informes y flujo completo)}

La generación de informes ocurre dentro de la función principal del archivo \textit{main.py},
que además gestiona el flujo completo de ejecución (desde la selección del tokenizador hasta la
escritura del informe final en la consola o en un archivo de salida).

Existen dos formatos principales de salida. Si el programa se ejecuta en modo interactivo (no
se proporciona un archivo de entrada), entonces se produce un informe simplificado. Este
formato está pensado para exploraciones rápidas o pruebas manuales. Por otro lado, si se
proporciona un archivo de texto para analizar, se realiza un informe más detallado.

El sistema también diferencia entre la salida por consola (donde se utilizan códigos ANSI para
resaltar visualmente los distintos tipos de tokens) y la salida a archivo, que se imprime sin
colores para facilitar su posterior procesamiento o revisión.

\section{Resultados}
El siguiente es el resultado del análisis de un ejemplo de conversación entre un agente y un
cliente. Se verifica que el agente ha cumplido con el protocolo de atención al cliente,
incluyendo saludo, identificación, y despedida, sin utilizar palabras prohibidas.

Además, las métricas y el análisis de sentimiento se realizan por separado para el cliente y el
agente, lo que facilita aún más la detección de discrepancias o valores atípicos en la
conversación.

\subsection{Caso de ejemplo}
\subsubsection*{Entrada:}
\begin{verbatim}
-----------------------------------------------------------------    
Agente: ¡Buenos días! Gracias por contactar con el servicio al 
cliente de ConexiónNet. Mi nombre es Johanna. ¿Cómo puedo 
ayudarle hoy?

Cliente: Hola, estoy teniendo algunos problemas con mi 
internet. Pago por 100 Mbps, pero últimamente he notado que 
la velocidad es mucho más baja.

Agente: Lamento escuchar eso, y estoy aquí para ayudarle 
a resolverlo. ¿Podría proporcionarme su nombre completo 
y el número de cuenta para que pueda revisar su situación?

Cliente: Claro, soy Juan Pérez y mi número de cuenta es 
12345678.

Agente: ¡Gracias, Juan! Voy a revisar su cuenta ahora mismo.
Veo que efectivamente tiene un plan de 100 Mbps. 
Déjeme hacer una prueba de línea para verificar cómo está 
funcionando su conexión en este momento.

Resto del texto ... 
-----------------------------------------------------------------    
\end{verbatim}

\subsubsection*{Salida:}
\begin{tcolorbox}[colback=gray!10, colframe=gray!80, sharp corners, boxrule=0.5pt]
	\begin{verbatim}
=== RESUMEN GENERAL ===
Puntaje total: -10
Sentimiento: NEGATIVO

=== CLIENTE ===
Puntaje: -6
Saludo: No
Despedida: Sí
Identificación: Sí
Palabras prohibidas: No

=== AGENTE ===
Puntaje: -4
Saludo: Sí
Despedida: Sí
Identificación: Sí
Palabras prohibidas: Sí

Nota: Se ignoraron las palabras desconocidas
          \end{verbatim}
\end{tcolorbox}

\subsubsection*{Tokens:}
\begin{tcolorbox}[colback=gray!10, colframe=gray!80, sharp corners, boxrule=0.5pt]
	\begin{verbatim}
	Valor                  Tipo                    Puntaje
	------------------------------------------------------
	agente:                TOKEN_AGENTE                  0
	Buenos días            TOKEN_SALUDO                  0
	,                      TOKEN_SIGNO_PUNTUACION        0
	¡                      TOKEN_SIGNO_PUNTUACION        0
	Gracias                TOKEN_SENTIMIENTO             1
	por                    TOKEN_DESCONOCIDO             0
	contactar              TOKEN_DESCONOCIDO             0
	con                    TOKEN_DESCONOCIDO             0
	el                     TOKEN_DESCONOCIDO             0

    Mas tokens...
\end{verbatim}
\end{tcolorbox}

\subsubsection*{AFD:}
Salida solo presente con el tokenizador de tipo AFD.
\begin{tcolorbox}[colback=gray!10, colframe=gray!80, sharp corners, boxrule=0.5pt]
    \begin{verbatim}
{
  "initial": "start",
  "states": {
    "start": {
      "transitions": {
        "e": [
          "start_e_sentimiento", 
          ...
          "start_e_saludo", 
          "start_e_despedida", 
          "start_e_prohibida"
        ],
        "m": [
          "start_m_sentimiento",
          ...
        ],
        ... mas transiciones
      },
      "is_final": true,
      "token_type": "prohibida",
      "puntuacion": -5
    },
    "start_e_sentimiento": {
      "transitions": {
        "x": [
          "start_e_sentimiento_x_sentimiento",
          ...
        ],
        ... mas transiciones
      },
      ... mas estados
    }
    ...
  }
}
\end{verbatim}
\end{tcolorbox}

\section{Conclusión}

Este proyecto implementa un sistema integral de análisis de sentimientos y verificación de
protocolos de atención mediante el uso de un Autómata Finito Determinista (AFD) y una tabla de
sentimientos. La arquitectura del sistema permite la identificación y clasificación de palabras
con carga emocional, así como la detección de palabras inapropiadas en las interacciones.
Además, el sistema verifica el cumplimiento de los elementos fundamentales del protocolo de
atención (saludo, identificación y despedida), proporcionando una evaluación detallada de las
respuestas del agente.

La flexibilidad del sistema para agregar o eliminar palabras de la tabla de sentimientos y
actualizar el AFD en tiempo real lo hace adaptable a nuevas necesidades o criterios específicos
de análisis. Esta capacidad de modificación y persistencia de datos asegura que el sistema se
pueda ajustar continuamente para mejorar la precisión de las evaluaciones y adaptarse a
diferentes contextos de uso.

En conjunto, este sistema representa una herramienta robusta y adaptable para el análisis
automatizado de interacciones en centros de contacto, con potencial de ser extendido o
integrado en aplicaciones más amplias de procesamiento de lenguaje natural (PLN) o de mejora de
calidad en la atención al cliente.


% Seccion de anexos
\section{Anexos}\label{sec:Anexos}
El codigo fuente completo de este proyecto puede ser encontrado en:
\href{https://github.com/elias-gill/tp-compiladores}{https://github.com/elias-gill/tp-compiladores}.
\subsection{Codigo fuente}

En esta seccion solamente se muestran las clases y componentes mas importantes del sistema.
Aquellas funciones utilitarias o componentes poco relevantes para el funcionamiento del
algoritmo de tokenizacion y analisis seran omitidos.

%%%%%%%%%%%%%%%%%%%%%%%%
%     CODIGO           %
%%%%%%%%%%%%%%%%%%%%%%%%
\subsubsection{Tabla de sentimientos}
\begin{lstlisting}
import difflib
from pathlib import Path

TIPO_SENTIMIENTO = "sentimiento"
TIPO_SALUDO = "saludo"
TIPO_DESPEDIDA = "despedida"
TIPO_IDENTIFICACION = "identificacion"
TIPO_PROHIBIDA = "prohibida"
TIPO_DESCONOCIDO = "desconocido"


class TablaSentimientos:
    # Constantes de archivos
    BASE_DIR = Path("tokenizer/sentiment_symbols")
    ARCHIVO_PUNTAJES = BASE_DIR / "palabras_y_puntajes.txt"
    ARCHIVO_SALUDOS = BASE_DIR / "saludos.txt"
    ARCHIVO_DESPEDIDAS = BASE_DIR / "despedidas.txt"
    ARCHIVO_IDENTIFICACIONES = BASE_DIR / "identificaciones.txt"
    ARCHIVO_PROHIBIDAS = BASE_DIR / "palabras_prohibidas.txt"

    def __init__(self):
        # Diccionarios separados
        self.palabras = {}
        self.saludos = {}
        self.despedidas = {}
        self.identificaciones = {}
        self.palabras_prohibidas = {}

        self._cargar_datos()

    def agregar_palabra(self, palabra, puntaje, persistir=True):
        palabra = palabra.lower().strip()
        self.palabras[palabra] = int(puntaje)
        if persistir:
            self._guardar_palabra_en_archivo(palabra, puntaje, self.ARCHIVO_PUNTAJES)

    def eliminar_palabra(self, palabra, persistir=True):
        palabra = palabra.lower().strip()
        if palabra in self.palabras:
            del self.palabras[palabra]
            if persistir:
                self._eliminar_palabra_de_archivo(palabra, self.ARCHIVO_PUNTAJES)
            return True
        return False

    def buscar_palabra(self, palabra):
        palabra = palabra.lower().strip()
        if palabra in self.saludos:
            return (TIPO_SALUDO, self.saludos[palabra])
        if palabra in self.despedidas:
            return (TIPO_DESPEDIDA, self.despedidas[palabra])
        if palabra in self.identificaciones:
            return (TIPO_IDENTIFICACION, self.identificaciones[palabra])
        if palabra in self.palabras_prohibidas:
            return (TIPO_PROHIBIDA, self.palabras_prohibidas[palabra])
        if palabra in self.palabras:
            return (TIPO_SENTIMIENTO, self.palabras[palabra])
        return (TIPO_DESCONOCIDO, 0)

    def _cargar_archivo_comun(self, archivo: Path, destino: dict):
        if not archivo.exists():
            return
        with open(archivo, "r", encoding="utf-8") as f:
            for linea in f:
                linea = linea.strip()
                if not linea:
                    continue
                try:
                    palabra, valor = linea.split(",", 1)
                    destino[palabra.lower()] = int(valor)
                except ValueError:
                    continue

    def _cargar_datos(self):
        self._cargar_archivo_comun(self.ARCHIVO_PUNTAJES, self.palabras)
        self._cargar_archivo_comun(self.ARCHIVO_SALUDOS, self.saludos)
        self._cargar_archivo_comun(self.ARCHIVO_DESPEDIDAS, self.despedidas)
        self._cargar_archivo_comun(self.ARCHIVO_IDENTIFICACIONES, self.identificaciones)
        self._cargar_archivo_comun(self.ARCHIVO_PROHIBIDAS, self.palabras_prohibidas)

    def _guardar_palabra_en_archivo(self, palabra, valor, archivo):
        try:
            with open(archivo, "a", encoding="utf-8") as f:
                f.write(f"{palabra},{valor}\n")
        except IOError as e:
            print(f"Error al guardar en archivo: {e}")

    def _eliminar_palabra_de_archivo(self, palabra, archivo):
        if not archivo.exists():
            return
        try:
            with open(archivo, "r", encoding="utf-8") as f:
                lineas = [
                    linea
                    for linea in f
                    if not linea.lower().startswith(f"{palabra.lower()},")
                ]
            with open(archivo, "w", encoding="utf-8") as f:
                f.writelines(lineas)
        except IOError as e:
            print(f"Error al modificar archivo: {e}")

    def sugerir_similares(self, palabra: str, max_sugerencias=3):
        palabra = palabra.lower().strip()

        # Palabras conocidas del sistema
        todas = (
            set(self.palabras)
            | set(self.saludos)
            | set(self.despedidas)
            | set(self.identificaciones)
            | set(self.palabras_prohibidas)
        )

        sugerencias = difflib.get_close_matches(
            palabra, todas, n=max_sugerencias, cutoff=0.6
        )
        return sugerencias
\end{lstlisting}

%%%%%%%%%%%%%%%%%%%%%%%%
%     CODIGO           %
%%%%%%%%%%%%%%%%%%%%%%%%
\subsubsection{Tokenizador AFD}
\begin{lstlisting}
import json
from pathlib import Path
from typing import Dict, List, Optional, Tuple

from tokenizer.TablaSentimientos import (TIPO_DESPEDIDA, TIPO_IDENTIFICACION,
                                         TIPO_PROHIBIDA, TIPO_SALUDO,
                                         TIPO_SENTIMIENTO, TablaSentimientos)
from tokenizer.tokens import (TOKEN_AGENTE, TOKEN_CLIENTE,
                              TOKEN_SIGNO_PUNTUACION, Token, asignar_tipo)


class AFDTokenizer:
    def __init__(self, tabla_sentimientos: TablaSentimientos):
        self.tabla = tabla_sentimientos
        self._build_afd_completo()
        self._persistir_afd()

    def _persistir_afd(self):
        """Guarda el AFD en un archivo JSON en la carpeta output"""
        output_dir = Path("output")
        output_dir.mkdir(exist_ok=True)

        afd_file = output_dir / "afd.json"
        with open(afd_file, "w", encoding="utf-8") as f:
            json.dump(self.afd, f, indent=2, ensure_ascii=False)

    def _build_afd_completo(self):
        self.afd = {
            "initial": "start",
            "states": {
                "start": {"transitions": {}, "is_final": False, "token_type": None}
            },
        }

        self._cargar_frases_al_afd(self.tabla.palabras, TIPO_SENTIMIENTO)
        self._cargar_frases_al_afd(self.tabla.saludos, TIPO_SALUDO)
        self._cargar_frases_al_afd(self.tabla.despedidas, TIPO_DESPEDIDA)
        self._cargar_frases_al_afd(self.tabla.identificaciones, TIPO_IDENTIFICACION)
        self._cargar_frases_al_afd(self.tabla.palabras_prohibidas, TIPO_PROHIBIDA)

    def _cargar_frases_al_afd(self, frases_dict: Dict[str, int], tipo: str):
        for frase in frases_dict.keys():
            chars = list(frase.lower())
            current_state = "start"

            for i, char in enumerate(chars):
                next_state = f"{current_state}_{char}_{tipo}"

                is_final = i == len(chars) - 1
                token_type = tipo if is_final else None
                puntuacion = frases_dict[frase] if is_final else 0

                if next_state not in self.afd["states"]:
                    self.afd["states"][next_state] = {
                        "transitions": {},
                        "is_final": is_final,
                        "token_type": token_type,
                        "puntuacion": puntuacion,
                    }

                if char not in self.afd["states"][current_state]["transitions"]:
                    self.afd["states"][current_state]["transitions"][char] = []
                self.afd["states"][current_state]["transitions"][char].append(
                    next_state
                )

                current_state = next_state

    def tokenizar(self, texto: str) -> List[Token]:
        tokens = []
        i = 0
        n = len(texto)

        texto = self._preprocesar_hablantes(texto)

        while i < n:
            if texto[i].isspace():
                i += 1
                continue

            if self._es_signo_puntuacion(texto[i]):
                tokens.append(Token(TOKEN_SIGNO_PUNTUACION, texto[i]))
                i += 1
                continue

            hablante_token = self._procesar_hablante(texto, i)
            if hablante_token:
                tokens.append(hablante_token)
                i += len(hablante_token.valor)
                continue

            token, delta = self._tokenizar_con_afd(texto, i)
            tokens.append(token)
            i += delta

        return tokens

    def _procesar_hablante(self, texto: str, start: int) -> Optional[Token]:
        if texto[start:].lower().startswith("agente:"):
            return Token(TOKEN_AGENTE, "agente:")
        elif texto[start:].lower().startswith("cliente:"):
            return Token(TOKEN_CLIENTE, "cliente:")
        return None

    def _preprocesar_hablantes(self, texto: str) -> str:
        import re

        return re.sub(
            r"\b(agente|cliente):",
            lambda m: f" {m.group(0)} ",
            texto,
            flags=re.IGNORECASE,
        )

    def _es_signo_puntuacion(self, char: str) -> bool:
        return not (char.isalnum() or char.isspace() or char in "'-_áéíóúüñ")

    def _tokenizar_con_afd(self, texto: str, start: int) -> Tuple[Token, int]:
        best_token = None
        best_length = 0
        current_states = [("start", "", 0)]

        i = start
        while i < len(texto) and current_states:
            char = texto[i].lower()
            new_states = []

            for state, acc, total_length in current_states:
                if char in self.afd["states"][state]["transitions"]:
                    for next_state in self.afd["states"][state]["transitions"][char]:
                        new_acc = acc + texto[i]
                        new_length = total_length + 1

                        if self.afd["states"][next_state]["is_final"]:
                            token_type = self.afd["states"][next_state]["token_type"]
                            puntuacion = self.afd["states"][next_state]["puntuacion"]

                            if new_length > best_length:
                                best_token = Token(
                                    asignar_tipo(token_type), new_acc, puntuacion
                                )
                                best_length = new_length

                        new_states.append((next_state, new_acc, new_length))

            current_states = new_states
            i += 1

        if best_token:
            return best_token, best_length

        return self._tokenizar_palabra_simple(texto, start)

    def _tokenizar_palabra_simple(self, texto: str, start: int) -> Tuple[Token, int]:
        i = start
        n = len(texto)

        while i < n and (texto[i].isalnum() or texto[i] in "'-_áéíóúüñ"):
            i += 1

        if i == start:
            return Token("DESCONOCIDO", texto[start]), 1

        palabra = texto[start:i].lower()
        tipo, puntuacion = self.tabla.buscar_palabra(palabra)
        return Token(asignar_tipo(tipo), palabra, puntuacion), i - start
\end{lstlisting}

%%%%%%%%%%%%%%%%%%%%%%%%
%     CODIGO           %
%%%%%%%%%%%%%%%%%%%%%%%%
\subsubsection{Tokenizador HashMap}
\begin{lstlisting}
import re

from tokenizer.TablaSentimientos import TIPO_DESCONOCIDO, TablaSentimientos
from tokenizer.tokens import (TOKEN_AGENTE, TOKEN_CLIENTE,
                              TOKEN_SIGNO_PUNTUACION, Token, asignar_tipo)


class HashTokenizer:
    def __init__(self, tabla_sentimientos: TablaSentimientos):
        self.tabla = tabla_sentimientos
        self._hablante_re = re.compile(r"\b(agente|cliente):", re.IGNORECASE)
        self._palabras_re = re.compile(r"\w+|[^\w\s]", re.UNICODE)

    def tokenizar(self, texto: str) -> list[Token]:
        # extraer los hablantes del texto
        texto = self._hablante_re.sub(lambda m: f" {m.group(0)} ", texto)
        palabras = self._palabras_re.findall(texto.lower())
        i = 0

        # tokenizar
        tokens = []
        while i < len(palabras):
            # Manejo de agente/cliente
            if i + 1 < len(palabras) and palabras[i + 1] == ":":
                if palabras[i] == "agente":
                    tokens.append(Token(TOKEN_AGENTE, "agente:"))
                    i += 2
                    continue
                if palabras[i] == "cliente":
                    tokens.append(Token(TOKEN_CLIENTE, "cliente:"))
                    i += 2
                    continue

            # Búsqueda de frases (3, 2, 1 palabras)
            for length in (3, 2, 1):
                if i + length > len(palabras):
                    continue

                frase = " ".join(palabras[i : i + length])
                tipo, puntuacion = self.tabla.buscar_palabra(frase)

                if tipo != TIPO_DESCONOCIDO:
                    tokens.append(Token(asignar_tipo(tipo), frase, puntuacion))
                    i += length
                    break
            else:  # Si no se encontró ninguna frase válida
                palabra = palabras[i]
                if re.match(r"\W", palabra):
                    tokens.append(Token(TOKEN_SIGNO_PUNTUACION, palabra))
                else:
                    tipo, puntuacion = self.tabla.buscar_palabra(palabra)
                    tokens.append(Token(asignar_tipo(tipo), palabra, puntuacion))
                i += 1

        return tokens
\end{lstlisting}

%%%%%%%%%%%%%%%%%%%%%%%%
%     CODIGO           %
%%%%%%%%%%%%%%%%%%%%%%%%
\subsubsection{Analizador de sentimientos}
\begin{lstlisting}
from dataclasses import dataclass, field
from typing import List

from tokenizer.TablaSentimientos import TablaSentimientos
from tokenizer.tokens import (TOKEN_DESCONOCIDO, TOKEN_DESPEDIDA,
                                   TOKEN_IDENTIFICACION, TOKEN_PROHIBIDA,
                                   TOKEN_SALUDO, Token)


@dataclass
class ResultadoParticipante:
    puntaje_total: int = 0
    hay_saludo: bool = False
    hay_despedida: bool = False
    hay_identificacion: bool = False
    hay_prohibidas: bool = False
    desconocidas: List[str] = field(default_factory=list)


@dataclass
class ResultadoConversacion:
    cliente: ResultadoParticipante
    agente: ResultadoParticipante
    puntaje_total: int = 0
    desconocidas_compartidas: List[str] = field(default_factory=list)


def manejar_palabra_desconocida(
    valor: str, tabla_sentimientos: TablaSentimientos
) -> int:
    """Muestra un menú ordenado para manejar una palabra desconocida."""

    print(f"\nPalabra desconocida: '{valor}'")

    while True:
        print("Opciones:")
        print("  [a] Agregar palabra manualmente")
        print("  [c] Corregir usando una sugerencia")
        print("  [i] Ignorar palabra")
        opcion = input("Seleccione una opción: ").strip().lower()

        if opcion == "a":
            try:
                puntaje = int(input(f"Ingrese puntaje para '{valor}': "))
                tabla_sentimientos.agregar_palabra(valor, puntaje)
                print(f"Palabra '{valor}' agregada con puntaje {puntaje}")
                return puntaje
            except ValueError:
                print("Entrada inválida. Ingrese un número entero.")

        elif opcion == "c":
            sugerencias = tabla_sentimientos.sugerir_similares(valor)
            if not sugerencias:
                print("No se encontraron sugerencias.")
                continue

            print("\nSugerencias:")
            for i, s in enumerate(sugerencias, 1):
                puntaje = tabla_sentimientos.buscar_palabra(s)[1]
                print(f"  {i}. {s} (puntaje: {puntaje})")

            seleccion = input("Seleccione el número de la sugerencia: ").strip()
            if seleccion.isdigit():
                idx = int(seleccion) - 1
                if 0 <= idx < len(sugerencias):
                    corregida = sugerencias[idx]
                    _, puntaje = tabla_sentimientos.buscar_palabra(corregida)
                    print(f" Usando '{corregida}' con puntaje {puntaje}")
                    return puntaje
                else:
                    print(" Número fuera de rango.")
            else:
                print("Entrada inválida.")

        elif opcion == "i":
            print(" Palabra ignorada.")
            return 0

        else:
            print(" Opción no válida.")


def analizar_sentimiento(
    tokens: List[Token], tabla_sentimientos: TablaSentimientos
) -> ResultadoConversacion:
    """Analiza los tokens diferenciando entre cliente y agente."""
    resultado = ResultadoConversacion(
        cliente=ResultadoParticipante(), agente=ResultadoParticipante()
    )
    palabras_desconocidas = []
    hablante_actual = "agente"  # 'cliente' o 'agente'

    # Primera pasada: procesar tokens conocidos
    for token in tokens:
        # Determinar hablante actual
        if token.type == "TOKEN_CLIENTE":
            hablante_actual = "cliente"
            continue
        elif token.type == "TOKEN_AGENTE":
            hablante_actual = "agente"
            continue

        # Si no hay hablante definido, saltar este token
        if not hablante_actual:
            continue

        # Obtener referencia al participante actual
        participante = getattr(resultado, hablante_actual)

        # Procesar token según tipo
        if token.type == TOKEN_PROHIBIDA:
            participante.hay_prohibidas = True
        elif token.type == TOKEN_SALUDO:
            participante.hay_saludo = True
        elif token.type == TOKEN_DESPEDIDA:
            participante.hay_despedida = True
        elif token.type == TOKEN_IDENTIFICACION:
            participante.hay_identificacion = True
        elif token.type == TOKEN_DESCONOCIDO:
            palabras_desconocidas.append((hablante_actual, token.valor))

        # Sumar puntuación siempre
        participante.puntaje_total += token.puntuacion
        resultado.puntaje_total += token.puntuacion

    # Segunda pasada: manejar palabras desconocidas
    if palabras_desconocidas:
        print("\nPalabras desconocidas encontradas:")
        for i, (_, palabra) in enumerate(palabras_desconocidas, 1):
            print(f"{i}. {palabra}")

        opcion = input("Desea corregir estas palabras (s/n): ").strip().lower()

        if opcion == "s":
            for hablante, palabra in palabras_desconocidas:
                participante = getattr(resultado, hablante)
                if puntaje := manejar_palabra_desconocida(palabra, tabla_sentimientos):
                    participante.puntaje_total += puntaje
                    resultado.puntaje_total += puntaje
                else:
                    participante.desconocidas.append(palabra)
        else:
            for hablante, palabra in palabras_desconocidas:
                getattr(resultado, hablante).desconocidas.append(palabra)

    return resultado
\end{lstlisting}

%%%%%%%%%%%%%%%%%%%%%%%%
%     CODIGO           %
%%%%%%%%%%%%%%%%%%%%%%%%
\subsubsection{Menu principal}
\begin{lstlisting}
"""
Dentro de este archivo se encuentra todo el codigo para realizar los reportes y manejar el cli.
En su mayoria es codigo irrelevante para el objetivo principal, que es el analisis de
sentimiento y especialmente el proceso de tokenizacion.
"""

import sys
from pathlib import Path

from tokenizer.AFDTokenizer import AFDTokenizer
from tokenizer.analisis import ResultadoConversacion, analizar_sentimiento
from tokenizer.HashTokenizer import HashTokenizer
from tokenizer.TablaSentimientos import TablaSentimientos

# Códigos ANSI para colores
RESET = "\033[0m"
BOLD = "\033[1m"
RED = "\033[31m"
GREEN = "\033[32m"
YELLOW = "\033[33m"
BLUE = "\033[34m"
MAGENTA = "\033[35m"
CYAN = "\033[36m"
WHITE = "\033[37m"

OUTPUT_PATH = Path("output")


def aplicar_color(texto, color, usar_color):
    return f"{color}{texto}{RESET}" if usar_color else texto


def imprimir_tokens(tokens, tabla_sentimientos=None, archivo=None, color=True):
    """Imprime los tokens en formato de tabla con opción de colores.

    Args:
        tokens: Lista de tokens a imprimir
        tabla_sentimientos: Opcional, para mostrar puntuaciones
        archivo: Opcional, archivo de salida (sin colores)
        color: Si es True, usa colores ANSI (solo en consola)
    """
    usar_color = color and archivo is None

    max_valor = max((len(token.valor) for token in tokens), default=10)
    max_tipo = max((len(token.type) for token in tokens), default=10)
    max_punt = 7  # ancho fijo para 'Puntaje'

    def col(text, color_code):
        return aplicar_color(text, color_code, usar_color)

    if tabla_sentimientos:
        encabezado = f"{col('Valor', BOLD):<{max_valor}}  {col('Tipo', BOLD):<{max_tipo}}  {col('Puntaje', BOLD):>{max_punt}}"
        ancho_linea = max_valor + 2 + max_tipo + 2 + max_punt
    else:
        encabezado = (
            f"{col('Valor', BOLD):<{max_valor}}  {col('Tipo', BOLD):<{max_tipo}}"
        )
        ancho_linea = max_valor + 2 + max_tipo

    linea_sep = "-" * ancho_linea

    # Para archivo, escribir con '\n'; para consola, print que ya lo agrega
    def out(linea):
        if archivo:
            archivo.write(linea + "\n")
        else:
            print(linea)

    out(encabezado)
    out(linea_sep)

    for token in tokens:
        valor = token.valor
        tipo = token.type
        puntuacion = getattr(token, "puntuacion", "")
        punt_str = f"{puntuacion:>{max_punt}}" if tabla_sentimientos else ""

        if archivo:
            linea = f"{valor:<{max_valor}}  {tipo:<{max_tipo}}"
            if tabla_sentimientos:
                linea += f"  {punt_str}"
            out(linea)
        else:
            linea = f"{col(valor, MAGENTA):<{max_valor}}  {col(tipo, CYAN):<{max_tipo}}"
            if tabla_sentimientos:
                linea += f"  {col(punt_str, YELLOW):>{max_punt}}"
            out(linea)


def si_no(cond, invertido=False, usar_colores=True):
    if not usar_colores:
        return "Sí" if cond else "No"
    color = GREEN if cond != invertido else RED
    return f"{color}Sí{RESET}" if cond else f"{color}No{RESET}"


def generar_seccion(nombre, datos, usar_colores=True):
    bold = BOLD if usar_colores else ""
    reset = RESET if usar_colores else ""
    header = f"\n{bold + CYAN if usar_colores else ''}=== {nombre.upper()} ==={reset}"
    return "\n".join(
        [
            header,
            f"{bold}Puntaje:{reset} {datos.puntaje_total}",
            f"{bold}Saludo:{reset} {si_no(datos.hay_saludo, usar_colores=usar_colores)}",
            f"{bold}Despedida:{reset} {si_no(datos.hay_despedida, usar_colores=usar_colores)}",
            f"{bold}Identificación:{reset} {si_no(datos.hay_identificacion, usar_colores=usar_colores)}",
            f"{bold}Palabras prohibidas:{reset} {si_no(datos.hay_prohibidas, invertido=True, usar_colores=usar_colores)}",
        ]
    )


def obtener_sentimiento(puntaje, usar_colores=True):
    if usar_colores:
        if puntaje > 0:
            return f"{GREEN}POSITIVO{RESET}"
        elif puntaje < 0:
            return f"{RED}NEGATIVO{RESET}"
        return f"{YELLOW}NEUTRAL{RESET}"
    return "POSITIVO" if puntaje > 0 else "NEGATIVO" if puntaje < 0 else "NEUTRAL"


def imprimir_resultados_analisis(resultado: ResultadoConversacion):
    OUTPUT_PATH.mkdir(exist_ok=True)
    # Archivo sin colores
    with open(OUTPUT_PATH / "reporte.txt", "w", encoding="utf-8") as f:
        f.write(
            "\n".join(
                [
                    "=== RESUMEN GENERAL ===",
                    f"Puntaje total: {resultado.puntaje_total}",
                    f"Sentimiento: {obtener_sentimiento(resultado.puntaje_total, usar_colores=False)}",
                    generar_seccion("Cliente", resultado.cliente, usar_colores=False),
                    generar_seccion("Agente", resultado.agente, usar_colores=False),
                    (
                        "\nNota: Se ignoraron las palabras desconocidas"
                        if resultado.cliente.desconocidas
                        or resultado.agente.desconocidas
                        else ""
                    ),
                ]
            )
        )

    # Consola con colores
    print(f"\n{BOLD}{MAGENTA}=== RESUMEN GENERAL ===")
    print(f"{BOLD}{BLUE}Puntaje total:{RESET} {resultado.puntaje_total}")
    print(
        f"{BOLD}{BLUE}Sentimiento:{RESET} {obtener_sentimiento(resultado.puntaje_total)}"
    )
    print(generar_seccion("Cliente", resultado.cliente))
    print(generar_seccion("Agente", resultado.agente))
    if resultado.cliente.desconocidas or resultado.agente.desconocidas:
        print(f"\n{YELLOW}Nota: Se ignoraron las palabras desconocidas{RESET}")


def procesar_archivo(archivo_entrada, tokenizador, tabla_sentimientos):
    try:
        try:
            texto = Path(archivo_entrada).read_text(encoding="utf-8")
        except FileNotFoundError:
            print(
                f"{RED}Error: {archivo_entrada} no encontrado.{RESET}",
                file=sys.stderr,
            )
            return False
        except PermissionError:
            print(
                f"{RED}Error: Sin permisos para leer {archivo_entrada}.{RESET}",
                file=sys.stderr,
            )
            return False

        print(f"\n{BOLD}{BLUE}Procesando:{RESET} {archivo_entrada}")

        tokens = tokenizador.tokenizar(texto)

        try:
            OUTPUT_PATH.mkdir(parents=True, exist_ok=True)
        except OSError as e:
            print(f"{RED}Error al crear {OUTPUT_PATH}: {e}{RESET}", file=sys.stderr)
            return False

        archivo_salida = OUTPUT_PATH / "tokens.txt"
        try:
            with open(archivo_salida, "w", encoding="utf-8") as f:
                imprimir_tokens(tokens, tabla_sentimientos, f, color=False)
        except OSError as e:
            print(
                f"{RED}Error al escribir {archivo_salida}: {e}{RESET}",
                file=sys.stderr,
            )
            return False

        print(
            f"{GREEN}Tokenización completada. Resultados en: {archivo_salida}{RESET}"
        )

        try:
            resultado = analizar_sentimiento(tokens, tabla_sentimientos)
            imprimir_resultados_analisis(resultado)
        except Exception as e:
            print(
                f"{YELLOW} Advertencia: Error en análisis de sentimientos: {e}{RESET}",
                file=sys.stderr,
            )
            return True  # éxito parcial

        return True
    except Exception as e:
        print(f"{RED} Error inesperado: {e}{RESET}", file=sys.stderr)
        return False


def modo_interactivo(tabla_sentimientos, tokenizador):
    print(f"\n{BOLD}{CYAN}=== Modo Interactivo ==={RESET}")
    print(f"Ingrese texto a analizar (escriba {YELLOW}salir{RESET} para terminar):")

    while True:
        texto = input(f"\n{BOLD}{BLUE}Texto:{RESET} ").strip()
        if texto.lower() == "salir":
            break
        if not texto:
            continue
        tokens = tokenizador.tokenizar(texto)
        print(f"\n{BOLD}{BLUE}Tokens encontrados:{RESET}")
        imprimir_tokens(tokens)
        imprimir_resultados_analisis(analizar_sentimiento(tokens, tabla_sentimientos))


def main():
    tabla_sentimientos = TablaSentimientos()
    usar_hashmap = "--hashmap" in sys.argv
    if usar_hashmap:
        tokenizador = HashTokenizer(tabla_sentimientos)
        sys.argv.remove("--hashmap")
    else:
        tokenizador = AFDTokenizer(tabla_sentimientos)

    if len(sys.argv) > 1:
        archivo = Path(sys.argv[1])
        if not archivo.exists():
            print(
                f"{RED} Error: El archivo {archivo} no existe.{RESET}", file=sys.stderr
            )
            sys.exit(1)
        sys.exit(0 if procesar_archivo(archivo, tokenizador, tabla_sentimientos) else 1)
    else:
        modo_interactivo(tabla_sentimientos, tokenizador)


if __name__ == "__main__":
    main()
\end{lstlisting}


\end{document}
