\documentclass[12pt,a4paper]{article}

% \usepackage[spanish, es-tabla]{babel}

% Fuentes y estilos
\usepackage[utf8]{inputenc}
\usepackage[version=3]{mhchem}
\usepackage[journal=jacs]{chemstyle}
\usepackage{amsmath}
\usepackage{amsfonts}
\usepackage{amssymb}
\usepackage{makeidx}
\usepackage[left=3cm,right=3cm,top=3cm,bottom=3cm]{geometry}

%Formato del títulos
\usepackage{titlesec}
\usepackage{enumitem}
\titleformat*{\section}{\bfseries\large}
\titleformat*{\subsection}{\bfseries\normalsize}

%%%% Entorno código - No usado de momento
\usepackage{tcolorbox}
\usepackage{textcomp}
\usepackage{listings}
\usepackage{xcolor}
\usepackage{verbatim}
\usepackage{listings}
\usepackage{xcolor}  % Opcional, para colores
\lstset{
    language=Python,
    basicstyle=\ttfamily\small,
    keywordstyle=\color{blue},
    stringstyle=\color{red},
    commentstyle=\color{gray},
    showstringspaces=false,
    breaklines=true
    extendedchars=true,
    inputencoding=utf8,
    literate=
    {á}{{\'a}}1
    {é}{{\'e}}1
    {í}{{\'i}}1
    {ó}{{\'o}}1
    {ú}{{\'u}}1
    {ñ}{{\~n}}1
    {ü}{{\"u}}1,
}

%%% Grafos
\usepackage{tikz}
\usetikzlibrary{automata, positioning, arrows}

%% Algoritmos 
\usepackage{algorithm}
\usepackage{algorithmic} 

% Imágenes
\usepackage{graphicx}
\usepackage{lmodern}

% Pies de pagina y referencias
\usepackage[stable]{footmisc}
\usepackage[section]{placeins}
\usepackage{fancyhdr}
\usepackage[colorlinks=true, linkcolor=black, urlcolor=blue, citecolor=black]{hyperref}

\begin{document}
%------------------------ Fin del preambulo -------------------------------

%%%%%%%%%%%%%%%%%%%%%%%%%%%%%%%%%%%%
%           CARATULA               %
%%%%%%%%%%%%%%%%%%%%%%%%%%%%%%%%%%%%
\begin{titlepage}
	\begin{center}

	% Logo institucional (opcional)
	% \includegraphics[width=0.4\textwidth]{logo_universidad.png}
	% \vspace*{1cm}

	% Título principal
	\vspace*{2cm} % Más espacio arriba del título
	{\LARGE \textbf{Análisis de Interacciones en Contact Centers} \\
		\vspace{0.3cm} % Espacio entre líneas del título
		\large con Speech Analytics y Tokenización \\}
	\vspace{1.5cm} % Espacio después del título

	% Autores (formato de dos columnas)
	\begin{minipage}{0.8\textwidth}
		\centering
		\large
		\textbf{Autores:} \\
		\vspace{0.5cm} % Espacio entre autores
		Elías Sebastián Gill Quintana \\
		\vspace{0.2cm} % Espacio entre autores
		% \textbf{Coautor:} \\
		Pablo Caballero [continar] \\ % Completa con el apellido
	\end{minipage}
	\vfill

	% Información académica
	\large
	\textbf{Trabajo Práctico} \\
	\vspace{0.3cm}
	\textbf{Asignatura:} Diseño de Compiladores \\
	\vspace{0.3cm}
	\textbf{Profesor:} Sergio Andrés Aranda Zemán \\
	\vspace{1cm}

	% Fecha
	\textbf{Mayo, 2025}
	\vspace*{1cm} % Espacio final

\end{center}

\end{titlepage}
\newpage

%%%%%%%%%%%%%%%%%%%%%%%%%%%%%%%%%%%%
%             Indice               %
%%%%%%%%%%%%%%%%%%%%%%%%%%%%%%%%%%%%

\tableofcontents
\newpage

%%%%%%%%%%%%%%%%%%%%%%%%%%%%%%%%%%%%
%      Inicio del documento        %
%%%%%%%%%%%%%%%%%%%%%%%%%%%%%%%%%%%%

% --- Secciones del documento --- 

\section*{\centering Resumen}
Este trabajo presenta un sistema para el análisis de interacciones en contact centers
utilizando técnicas de Speech Analytics y tokenización. El objetivo principal es procesar
transcripciones de audio, clasificando palabras en categorías como saludos, despedidas,
identificaciones y términos con connotación emocional, además de realizar análisis de
sentimientos. 

La implementación base utiliza un autómata finito determinista (AFD) para el reconocimiento de
patrones conversacionales y verificación de protocolos de atención. Como contribución adicional
derivada del tiempo disponible, hemos desarrollado una segunda versión intercambiable del
tokenizador basada en tablas hash, permitiendo comparar ambos enfoques en términos de
eficiencia y precisión.

\section{Introducción}

En el contexto de los contact centers, donde la calidad de la interacción es crucial,
presentamos un sistema con dos implementaciones modularizadas de tokenización.

El núcleo del sistema emplea un Autómata Finito Determinista (AFD) capaz de identificar
estructuras conversacionales complejas, como secuencias de saludo, verificar el cumplimiento de
protocolos de servicio y facilitar su extensión mediante la adición de nuevos estados.

Como desarrollo complementario, se implementó una versión intercambiable basada en diccionarios
hash, diseñada para búsquedas léxicas de tiempo constante y optimizada para el procesamiento
eficiente de términos sueltos. Esta versión comparte la misma interfaz que el tokenizador
basado en AFD, lo que permite alternar entre ambas sin alterar el resto del sistema.

Ambas implementaciones pueden seleccionarse en tiempo de ejecución, lo cual demuestra cómo un
mismo problema puede abordarse mediante paradigmas computacionales distintos, pero equivalentes
en funcionalidad. En todos los casos, el sistema conserva su capacidad para clasificar
semánticamente segmentos como saludos y despedidas, aplicar análisis de sentimientos mediante
puntuación léxica y generar métricas sobre la calidad del servicio.

Este diseño dual no solo enriquece el valor académico del proyecto, sino que además permite
obtener evidencia empírica sobre las ventajas comparativas de cada enfoque en distintos
escenarios operativos. Los detalles de implementación se presentan en la Sección 4.
 % y resumen
\section{Arquitectura}

El sistema cuenta con los siguientes módulos:

\begin{itemize}
	\item Tabla de sentimientos
	\item Tokenizador
	\item Analizador de sentimientos
	\item Reportes (menú principal)
\end{itemize}

Cada modulo esta diseñado para realizar funciones especificas de manera modular, por lo que
se pueden realizar distintas implementaciones siempre y cuando se sigan las interfaces comunes.

\subsection{Flujo del programa}
El programa inicia cargando la tabla con las palabras necesarias para realizar el análisis de
sentimientos. A continuación, se selecciona e instancia el tipo de tokenizador, según los
argumentos proporcionados al momento de ejecutar el programa.

Si se indica un archivo como argumento, su contenido es procesado automáticamente. En caso
contrario, el programa entra en modo interactivo, permitiendo al usuario ingresar texto
directamente por consola.

La entrada es primero tokenizada utilizando el tokenizador seleccionado. Con la lista completa
de tokens disponibles, aquellos que son clasificados como \texttt{desconocidos} son presentados
al usuario para su corrección o descarte. Para cada token desconocido, el usuario puede elegir
entre las siguientes opciones:

\begin{itemize}
	\item Ignorar el token.
	\item Añadirlo a la tabla de sentimientos.
	\item Corregirlo utilizando una de las palabras sugeridas (si existen).
\end{itemize}

En caso de optar por la corrección mediante sugerencias, se muestra una lista de palabras
similares, calculadas utilizando la distancia de Levenshtein.

Finalmente, se realiza el análisis de sentimiento utilizando los tokens generados. El resultado
del análisis se muestra por consola y también se guarda en un archivo de texto plano como
resumen del procesamiento. También se almacena la lista de tokens resultantes del proceso de
tokenizacion.

\subsection{Tabla de Sentimientos}
La tabla de sentimientos es una estructura central del sistema, que contiene un conjunto de
palabras y frases categorizadas por tipo (por ejemplo: positivas, negativas, neutras) junto con
los puntajes asociados a cada una. Estos puntajes son utilizados posteriormente para calcular
la polaridad del texto analizado.

Además de su rol como base de conocimiento, esta tabla incluye funcionalidades de conveniencia
que facilitan su uso y actualización dinámica durante el análisis. Entre las capacidades
principales se destacan:

\begin{itemize}
	\item \textbf{Búsqueda de palabras o frases:} permite consultar rápidamente si una palabra
	      o frase está presente en la tabla y recuperar su puntaje asociado.

	\item \textbf{Cálculo de correcciones recomendadas:} Calcula sugerencias de posibles
	      correcciones utilizando la distancia de Levenshtein, proponiendo palabras similares ya
	      registradas.

	\item \textbf{Incorporación de nuevas entradas:} el usuario puede añadir palabras con
	      su respectivo puntaje de sentimiento, ampliando así progresivamente la tabla a medida
	      que se utiliza el programa.
\end{itemize}

\subsection{Tokenizador}
El tokenizador es el componente encargado de segmentar el texto de entrada en unidades léxicas
denominadas \emph{tokens}. Es capaz de identificar y clasificar distintos tipos de tokens,
tales como palabras simples, frases compuestas, signos de puntuación, números o símbolos
especiales. Además, posee mecanismos para resolver ambigüedades en aquellos casos donde
múltiples tipos pudieran coincidir con un mismo patrón textual.

Este proceso no se limita a una separación mecánica por espacios o signos, sino que aplica
reglas específicas para reconocer construcciones con significado semántico relevante para el
análisis de sentimientos.

El tokenizador recibe una entrada en forma de texto plano y retorna una lista estructurada de
tokens.

\subsubsection{Tokens}
Cada token incluye la siguiente información:

\begin{itemize}
	\item \textbf{Texto del token:} la representación literal extraída del texto original.
	\item \textbf{Tipo:} categoría asignada al token (por ejemplo, palabra conocida, palabra desconocida, puntuación, etc.).
	\item \textbf{Puntaje:} valor numérico asociado al token, obtenido de la tabla de sentimientos si corresponde.
\end{itemize}

\subsection{Analisis de sentimiento}

Una vez finalizado el proceso de tokenización, el sistema obtiene la lista completa de tokens
extraídos del texto. A continuación, se procede al análisis de estos tokens para determinar su
validez y relevancia en el contexto del análisis de sentimientos.

Si se detectan tokens de tipo \emph{desconocido} (que no se encuentran registrados en la tabla
de sentimientos), el sistema solicita al usuario elegir si ignorarlos o proceder a la
corrección manual de dichos tokens. En caso de que el usuario decida corregir los tokens
desconocidos, este módulo se encarga de gestionar todo el proceso de corrección.

El análisis de sentimiento se realiza de forma diferenciada para cada hablante, lo que permite
evaluar en detalle el desempeño individual del cliente y del agente dentro de la conversación.

\subsection{Reportes}
El módulo de reportes se encarga de generar archivos con los resultados del proceso de
tokenización y del análisis de sentimiento. Además, presenta un resumen conciso directamente en
la terminal.

%%%%%%%%%%%%%%%%%%%%%%%%%%%%%%%%%%%%%%%%%%%%%%%%%
% Decisiones clave en la parte de implemetacion %
%%%%%%%%%%%%%%%%%%%%%%%%%%%%%%%%%%%%%%%%%%%%%%%%%
\subsection{Decisiones clave}

\subsubsection{Tipos de tokens}
Los tipos de tokens identificados por el tokenizador son:

\begin{enumerate}
	\item {\footnotesize\textbf{TOKEN\_SALUDO}}: frase o palabra que corresponde a un saludo.
	\item {\footnotesize\textbf{TOKEN\_DESPEDIDA}}: frase o palabra de despedida.
	\item {\footnotesize\textbf{TOKEN\_IDENTIFICACION}}: indica la identificación de alguna de las partes.
	\item {\footnotesize\textbf{TOKEN\_PROHIBIDA}}: palabra o frase grosera.
	\item {\footnotesize\textbf{TOKEN\_SENTIMIENTO}}: palabra individual con un sentimiento asociado.
	\item {\footnotesize\textbf{TOKEN\_DESCONOCIDO}}: palabra no registrada en la tabla de sentimientos.
	\item {\footnotesize\textbf{TOKEN\_AGENTE}}: indica cambio de hablante a un agente.
	\item {\footnotesize\textbf{TOKEN\_CLIENTE}}: indica cambio de hablante a un cliente.
	\item {\footnotesize\textbf{TOKEN\_SIGNO\_PUNTUACION}}: representa signos de puntuación.
\end{enumerate}

\subsubsection{Detección de frases}
Como parte del diseño del sistema, se asumió el desafío de identificar no solo palabras
aisladas, sino también frases cortas como \textit{"buenos días"}, con el fin de facilitar la
identificación de protocolos como saludos o identificaciones.

Esto implica que el sistema debe resolver ambigüedades, por ejemplo, entre \textit{"buenos"}
(sentimiento) y \textit{"buenos días"} (saludo).

Esto también influye directamente en la elección del método de tokenización, ya que debe
priorizar correctamente las frases completas por sobre palabras sueltas.

\subsubsection{API de la tabla de sentimientos}
La tabla de sentimientos almacena los distintos tipos de frases (saludos, sentimientos, etc.)
en diccionarios separados. Además, cuenta con funciones auxiliares, entre ellas la más
importante es \textbf{\textit{buscar\_palabra()}}, utilizada especialmente por el tokenizador
basado en hashmaps.

Esta función garantiza la siguiente prioridad al buscar una palabra:

\begin{enumerate}
	\item \footnotesize{TOKEN\_SALUDO}
	\item \footnotesize{TOKEN\_DESPEDIDA}
	\item \footnotesize{TOKEN\_IDENTIFICACION}
	\item \footnotesize{TOKEN\_PROHIBIDA}
	\item \footnotesize{TOKEN\_SENTIMIENTO}
	\item \footnotesize{TOKEN\_DESCONOCIDO}
\end{enumerate}

Por lo tanto, si una palabra aparece en múltiples categorías, se prioriza el tipo más
específico (como saludo, despedida, etc.) antes que un simple sentimiento.

\subsubsection{Mecanismo de tokenizacion}
Dado el tiempo disponible, se desarrollaron dos tipos de tokenizadores: uno basado en autómatas
finitos deterministas (AFD) y otro basado en tablas hash.

Ambos implementan una interfaz común con el método \textbf{\textit{tokenizar()}}, el cual
devuelve una lista con los tokens analizados.

Se consideró la posibilidad de utilizar un método \textbf{\textit{next\_token()}}, pero se optó
por una solución más simple que retorne todos los tokens de una vez.

Gracias a su diseño modular, los tokenizadores son fácilmente intercambiables, lo que permite
realizar comparaciones sobre facilidad de mantenimiento, extensibilidad, implementación y
rendimiento.

\subsubsection{Solución de ambigüedades}
En el tokenizador basado en hashmaps, se emplea un buffer de hasta 3 palabras, lo que permite
dar prioridad a las frases más largas. Si no se encuentra una coincidencia con 3 palabras, se
intenta con 2 y luego con 1.

En el caso del tokenizador AFD se utiliza el algoritmo de \textit{longest match}, es decir, se
continúa la lectura mientras el autómata permita avanzar hacia un estado válido, favoreciendo
la identificación de frases completas. La lógica es equivalente a la usada en el tokenizador
hash, respetando la misma prioridad de carga:

\begin{verbatim}
self._cargar_frases_al_afd(self.tabla.saludos, TIPO_SALUDO)
self._cargar_frases_al_afd(self.tabla.despedidas, TIPO_DESPEDIDA)
self._cargar_frases_al_afd(self.tabla.identificaciones, TIPO_IDENTIFICACION)
self._cargar_frases_al_afd(self.tabla.palabras_prohibidas, TIPO_PROHIBIDA)
self._cargar_frases_al_afd(self.tabla.palabras, TIPO_SENTIMIENTO)
\end{verbatim}

Así, si una palabra aparece en más de un conjunto, se considera como "canónica" aquella
asociada al tipo más específico (es decir, distinto de sentimiento). Más detalles sobre el
algoritmo de "longest match" y el proceso de carga del AFD se explican en las secciones
siguientes.

\section{Implementacion}
\textit{NOTA:} aqui se explican la implementacion de los modulos principales del proyecto, pero
el codigo fuente se encuentra en las seccion de anexos.

\subsection{Estructura del codigo}
% TODO: agregar una descripcion de las clases principales
El sistema está estructurado de la siguiente manera:

\begin{tcolorbox}[colback=gray!10, colframe=gray!80, sharp corners, boxrule=0.5pt]
	\begin{verbatim}
|-- main.py
|-- output
|   `-- reportes ...
|-- test
|   `-- textos de prueba ...
`-- tokenizer
    |-- AFDTokenizer.py
    |-- analisis.py
    |-- HashTokenizer.py
    |-- sentiment_symbols
    |   |-- despedidas.txt
    |   |-- identificaciones.txt
    |   |-- palabras_prohibidas.txt
    |   |-- palabras_y_puntajes.txt
    |   `-- saludos.txt
    |-- TablaSentimientos.py
    `-- tokens.py
\end{verbatim}
\end{tcolorbox}

%%%%%%%%%%%%%%%%%%%%%%%%%%%%%%%%%%%
%    La tabla de sentimientos     %
%%%%%%%%%%%%%%%%%%%%%%%%%%%%%%%%%%%
\subsubsection{Tabla de Sentimientos}
\texttt{TablaSentimientos} es una clase encargada de almacenar, gestionar y consultar
diferentes tipos de palabras utilizadas durante el análisis de sentimiento. Para ello, carga en
memoria cinco diccionarios distintos desde archivos persistentes: palabras con puntaje,
saludos, despedidas, identificaciones y palabras prohibidas, a través del método \texttt{\_cargar\_datos()}.

Permite agregar o eliminar palabras con su puntaje asociado mediante los métodos
\texttt{agregar\_palabra()} y \texttt{eliminar\_palabra()}, modificando tanto la estructura en
memoria como los archivos de respaldo en disco.

Además, provee un método de búsqueda, \texttt{buscar\_palabra()}, que clasifica una palabra
dada en uno de los tipos conocidos (como saludo, despedida o sentimiento) o la marca como
desconocida si no está registrada.

La clase ofrece un sistema de sugerencias automáticas basado en la similitud de cadenas
(\texttt{difflib.get\_close\_matches}), accesible mediante el método \texttt{sugerir\_similares()},
facilitando así la corrección y actualización interactiva de la tabla de sentimientos.

%%%%%%%%%%%%%%%%%%%%%%%%%%%%%%%%%
%      Los tokenizadores        %
%%%%%%%%%%%%%%%%%%%%%%%%%%%%%%%%%
\subsection{Tokenizacion}

%%%%%  Tokenizador AFD  %%%%%
\subsubsection{Tokenizador AFD}

La clase \texttt{AFDTokenizer} transforma un texto de entrada en una secuencia de tokens
semánticamente significativos. Para ello, utiliza un autómata finito determinista (AFD)
generado a partir de una instancia de \texttt{TablaSentimientos}, y aplica un algoritmo de
\textit{longest match} para asegurar coincidencias máximas.

\subsubsection{Construcción del AFD}

El AFD se inicializa mediante el método \texttt{\_build\_afd\_completo()}, que crea un estado
inicial denominado \texttt{start} y lo extiende agregando transiciones según los distintos
tipos de frases registrados en \texttt{TablaSentimientos}. Estas categorías incluyen:

\begin{itemize}
	\item Palabras con puntaje (\texttt{self.tabla.palabras})
	\item Saludos
	\item Despedidas
	\item Identificaciones
	\item Palabras prohibidas
\end{itemize}

La incorporación se realiza con \texttt{\_cargar\_frases\_al\_afd()}, que descompone cada frase
carácter por carácter. Se generan estados intermedios hasta llegar al último carácter, cuyo
estado asociado se marca como final. En dicho estado final se almacenan el tipo de token y, si
corresponde, el puntaje asociado.

\subsubsection{Persistencia del AFD}

El método \texttt{\_persistir\_afd()} guarda la estructura resultante en formato JSON bajo la
carpeta \texttt{output}, facilitando su inspección externa o reutilización sin reconstrucción.

\subsubsection{Preprocesamiento del texto}

Antes de tokenizar, se aplica \texttt{\_preprocesar\_hablantes()} para normalizar marcas como
\texttt{agente:} o \texttt{cliente:}, asegurando que sean detectadas como unidades separadas.
Luego, \texttt{\_procesar\_hablante()} se encarga de convertir estas marcas en tokens
especializados.

\subsubsection{Tokenización con AFD y \textit{longest match}}

El núcleo del análisis léxico ocurre en \texttt{tokenizar()}, que recorre el texto carácter por
carácter. Cuando se identifica una palabra candidata, se invoca
\texttt{\_tokenizar\_con\_afd()}.

Este método implementa un algoritmo de \textbf{búsqueda por mayor coincidencia}
(\textit{longest match}) sobre el AFD. A partir de la posición actual, se simulan todas las
transiciones posibles en paralelo:

\begin{itemize}
	\item Cada vez que se alcanza un estado final, se guarda un token candidato.
	\item Se prioriza el token que consuma la mayor cantidad de caracteres consecutivos desde la posición inicial.
	\item Si hay múltiples candidatos válidos, se conserva sólo el más largo.
\end{itemize}

Esto permite, por ejemplo, reconocer \texttt{muy buen servicio} como una sola frase si está
registrada, en lugar de dividirla en partes.

\subsubsection{Fallback: \texttt{\_tokenizar\_palabra\_simple()}}

Si \texttt{\_tokenizar\_con\_afd()} no logra reconocer ningún token válido, se recurre al
método \texttt{\_tokenizar\_palabra\_simple()}. Este método agrupa letras consecutivas para
formar una palabra y consulta \texttt{TablaSentimientos.buscar\_palabra()}. Si la palabra no
está registrada, se clasifica como \texttt{DESCONOCIDO}.

\subsubsection{Manejo de signos de puntuación}

Cualquier carácter que no sea alfanumérico ni parte de una palabra compuesta se clasifica como
signo de puntuación mediante \texttt{\_es\_signo\_puntuacion()}, y se encapsula en un token con
tipo \texttt{TOKEN\_SIGNO\_PUNTUACION}.

\subsubsection{Resumen del flujo de análisis léxico}

\begin{enumerate}
	\item Se preprocesa el texto para separar marcas de hablante.
	\item Se recorre el texto carácter a carácter.
	\item Se tokenizan signos de puntuación y hablantes de forma directa.
	\item Se intenta reconocer frases usando el AFD y \texttt{\_tokenizar\_con\_afd()} con \textit{longest match}.
	\item Si el AFD falla, se analiza como palabra suelta con \texttt{\_tokenizar\_palabra\_simple()}.
\end{enumerate}

Este enfoque garantiza flexibilidad y precisión: reconoce expresiones compuestas, clasifica
palabras sueltas, y marca adecuadamente lo desconocido, manteniendo una estructura extensible y
fácil de actualizar mediante los diccionarios gestionados por \texttt{TablaSentimientos}.

%%%%%  Tokenizador hashmap  %%%%%
\subsection{Tokenizador Hashmap}

El tokenizador \texttt{HashTokenizer} ofrece una alternativa más simple y directa al uso de un
autómata finito determinista (AFD). Su funcionamiento se basa en una exploración secuencial del
texto con búsqueda de coincidencias en un diccionario hash (la instancia de
\texttt{TablaSentimientos}). Aunque menos sofisticado que el AFD, su implementación es rápida y
efectiva para corpus de tamaño moderado.

\subsubsection{Expresiones regulares}

Se utilizan dos expresiones regulares precompiladas para facilitar el preprocesamiento del
texto:

\begin{itemize}
	\item \texttt{\_hablante\_re}: Detecta marcas de hablante como \texttt{agente:} o
	      \texttt{cliente:} para separarlas del resto del texto.

	\item \texttt{\_palabras\_re}: Segmenta el texto en unidades léxicas válidas. Detecta
	      palabras alfanuméricas y signos de puntuación como unidades separadas.
\end{itemize}

\subsubsection{Algoritmo de tokenización}

El método \texttt{tokenizar()} transforma el texto en una lista de tokens mediante los
siguientes pasos:

\begin{enumerate}
	\item Se normalizan las marcas de hablante agregando espacios alrededor de \texttt{agente:}
	      y \texttt{cliente:}.

	\item Se aplica la expresión regular para obtener una lista de palabras y signos.

	\item Se recorre la lista desde la izquierda aplicando una búsqueda decreciente de frases
	      de longitud 3, 2 y 1 palabras.
\end{enumerate}

\subsubsection{Manejo de marcas de hablante}

Cuando el patrón detecta secuencias como \texttt{agente :} o \texttt{cliente :}, se agrupan y
transforman en un token específico (\texttt{TOKEN\_AGENTE} o \texttt{TOKEN\_CLIENTE}). Este
paso se prioriza antes de intentar reconocer frases.

\subsubsection{Reconocimiento de frases por ventana deslizante}

En cada posición, se intenta formar una frase de longitud 3, luego 2, y finalmente 1 palabra.
Para cada combinación posible:

\begin{itemize}
	\item Se forma la frase concatenando las palabras con espacios.

	\item Se consulta \texttt{tabla.buscar\_palabra(frase)}.

	\item Si se encuentra una coincidencia, se genera un token con el tipo correspondiente y se
	      avanza el cursor \texttt{i} en la longitud de la frase.
\end{itemize}

Este mecanismo puede considerarse una versión simplificada del algoritmo de \textit{longest
	match}, sin necesidad de una estructura AFD.

\subsubsection{Fallback: palabra o signo aislado}

Si no se reconoce ninguna frase:

\begin{itemize}
	\item Si el fragmento es un carácter no alfanumérico (\texttt{\textbackslash W}), se
	      clasifica como \texttt{TOKEN\_SIGNO\_PUNTUACION}.

	\item Si es una palabra, se vuelve a consultar en la tabla de sentimientos como palabra
	      individual. Si no se encuentra, se clasifica como \texttt{DESCONOCIDO}.
\end{itemize}

\subsubsection{Ventajas y limitaciones}

\textbf{Ventajas:}

\begin{itemize}
	\item Implementación más directa y comprensible.
	\item Menor sobrecarga computacional inicial: no requiere construcción previa de un AFD.
	\item Razonablemente eficaz cuando las frases significativas tienen longitud acotada.
\end{itemize}

\textbf{Limitaciones:}

\begin{itemize}
	\item No garantiza coincidencias máximas globales, ya que detiene la búsqueda en la primera
	      coincidencia de mayor longitud encontrada.

	\item Puede ser menos eficiente para grandes volúmenes de texto o diccionarios más
	      extensos.
\end{itemize}

Este tokenizador ofrece un buen equilibrio entre simplicidad y cobertura, y resulta útil como
referencia base o método de respaldo frente a sistemas más complejos como el basado en AFD.

%%%%%%%%%%%%%%%%%%%%%%%%%%%%%%%%%%%%%%%%%
%      Analizador de sentimiento        %
%%%%%%%%%%%%%%%%%%%%%%%%%%%%%%%%%%%%%%%%%
\subsection{Análisis de sentimiento y protocolo de atención}

El análisis de sentimiento se basa en recorrer la secuencia de tokens de la conversación,
asignando puntajes y detectando eventos clave para cliente y agente.

Se distingue quién habla en cada momento y se acumulan las métricas asociadas. También se
identifican eventos importantes para evaluar el protocolo de atención.

\subsubsection{Protocolos y etiquetas especiales}

Se detectan eventos que afectan la evaluación del protocolo:

\begin{itemize}
	\item \textbf{Saludo}: apertura cordial.
	\item \textbf{Despedida}: cierre adecuado.
	\item \textbf{Identificación}: presentación del agente.
	\item \textbf{Palabras prohibidas}: lenguaje inapropiado.
\end{itemize}

\subsubsection{Estructura del procesamiento}

La función principal recorre los tokens para:

\begin{itemize}
	\item Identificar el hablante actual (cliente o agente).
	\item Acumular puntajes y marcar eventos por hablante.
	\item Detectar palabras desconocidas para su posterior manejo.
\end{itemize}

\begin{verbatim}
def analizar_sentimiento(tokens, tabla_sentimientos):
    resultado = ResultadoConversacion(...)
    palabras_desconocidas = []
    hablante_actual = "agente"

    for token in tokens:
        if token.type == "TOKEN_CLIENTE":
            hablante_actual = "cliente"
            continue
        elif token.type == "TOKEN_AGENTE":
            hablante_actual = "agente"
            continue

        participante = getattr(resultado, hablante_actual)

        if token.type == TOKEN_PROHIBIDA:
            participante.hay_prohibidas = True
        elif token.type == TOKEN_SALUDO:
            participante.hay_saludo = True
        elif token.type == TOKEN_DESPEDIDA:
            participante.hay_despedida = True
        elif token.type == TOKEN_IDENTIFICACION:
            participante.hay_identificacion = True
        elif token.type == TOKEN_DESCONOCIDO:
            palabras_desconocidas.append((hablante_actual, token.valor))

        participante.puntaje_total += token.puntuacion
        resultado.puntaje_total += token.puntuacion
\end{verbatim}

Luego, si hay palabras desconocidas, se las maneja según se explica a continuación.

\subsubsection{Corrección de tokens desconocidos}

Durante el análisis, las palabras no encontradas en la tabla de sentimientos se clasifican como
desconocidas. Para cada una se ofrece un menú interactivo con opciones:

\begin{itemize}
	\item \textbf{Agregar manualmente}: ingresar puntaje para agregar la palabra.
	\item \textbf{Corregir con sugerencia}: elegir palabra similar existente.
	\item \textbf{Ignorar}: no modificar el puntaje.
\end{itemize}

\begin{verbatim}
[a] Agregar palabra manualmente")
[c] Corregir usando una sugerencia")
[i] Ignorar palabra")
    Seleccione una opción:
\end{verbatim}

Si el usuario decide agregar o corregir, el puntaje se suma al hablante y a la conversación. Si
ignora, se registra la palabra para análisis futuro.

Este procedimiento permite mejorar y ajustar dinámicamente la tabla de sentimientos para
futuros análisis.

%%%%%%%%%%%%%%%%%%%%%%%%%%%%%%
%      Menu principal        %
%%%%%%%%%%%%%%%%%%%%%%%%%%%%%%
\subsection{Generacion de informes}

\section{Resultados}
El siguiente es el resultado del análisis de un ejemplo de conversación entre un agente y un
cliente. Se verifica que el agente ha cumplido con el protocolo de atención al cliente,
incluyendo saludo, identificación, y despedida, sin utilizar palabras prohibidas.

Además, las métricas y el análisis de sentimiento se realizan por separado para el cliente y el
agente, lo que facilita aún más la detección de discrepancias o valores atípicos en la
conversación.

\subsection{Caso de ejemplo}
\subsubsection*{Entrada:}
\begin{verbatim}
-----------------------------------------------------------------    
Agente: ¡Buenos días! Gracias por contactar con el servicio al 
cliente de ConexiónNet. Mi nombre es Johanna. ¿Cómo puedo 
ayudarle hoy?

Cliente: Hola, estoy teniendo algunos problemas con mi 
internet. Pago por 100 Mbps, pero últimamente he notado que 
la velocidad es mucho más baja.

Agente: Lamento escuchar eso, y estoy aquí para ayudarle 
a resolverlo. ¿Podría proporcionarme su nombre completo 
y el número de cuenta para que pueda revisar su situación?

Cliente: Claro, soy Juan Pérez y mi número de cuenta es 
12345678.

Agente: ¡Gracias, Juan! Voy a revisar su cuenta ahora mismo.
Veo que efectivamente tiene un plan de 100 Mbps. 
Déjeme hacer una prueba de línea para verificar cómo está 
funcionando su conexión en este momento.

Resto del texto ... 
-----------------------------------------------------------------    
\end{verbatim}

\subsubsection*{Salida:}
\begin{tcolorbox}[colback=gray!10, colframe=gray!80, sharp corners, boxrule=0.5pt]
	\begin{verbatim}
=== RESUMEN GENERAL ===
Puntaje total: -10
Sentimiento: NEGATIVO

=== CLIENTE ===
Puntaje: -6
Saludo: No
Despedida: Sí
Identificación: Sí
Palabras prohibidas: No

=== AGENTE ===
Puntaje: -4
Saludo: Sí
Despedida: Sí
Identificación: Sí
Palabras prohibidas: Sí

Nota: Se ignoraron las palabras desconocidas
          \end{verbatim}
\end{tcolorbox}

\subsubsection*{Tokens:}
\begin{tcolorbox}[colback=gray!10, colframe=gray!80, sharp corners, boxrule=0.5pt]
	\begin{verbatim}
	Valor                  Tipo                    Puntaje
	------------------------------------------------------
	agente:                TOKEN_AGENTE                  0
	Buenos días            TOKEN_SALUDO                  0
	,                      TOKEN_SIGNO_PUNTUACION        0
	¡                      TOKEN_SIGNO_PUNTUACION        0
	Gracias                TOKEN_SENTIMIENTO             1
	por                    TOKEN_DESCONOCIDO             0
	contactar              TOKEN_DESCONOCIDO             0
	con                    TOKEN_DESCONOCIDO             0
	el                     TOKEN_DESCONOCIDO             0

    Mas tokens...
\end{verbatim}
\end{tcolorbox}

\subsubsection*{AFD:}
Salida solo presente con el tokenizador de tipo AFD.
\begin{tcolorbox}[colback=gray!10, colframe=gray!80, sharp corners, boxrule=0.5pt]
    \begin{verbatim}
{
  "initial": "start",
  "states": {
    "start": {
      "transitions": {
        "e": [
          "start_e_sentimiento", 
          ...
          "start_e_saludo", 
          "start_e_despedida", 
          "start_e_prohibida"
        ],
        "m": [
          "start_m_sentimiento",
          ...
        ],
        ... mas transiciones
      },
      "is_final": true,
      "token_type": "prohibida",
      "puntuacion": -5
    },
    "start_e_sentimiento": {
      "transitions": {
        "x": [
          "start_e_sentimiento_x_sentimiento",
          ...
        ],
        ... mas transiciones
      },
      ... mas estados
    }
    ...
  }
}
\end{verbatim}
\end{tcolorbox}

\section{Conclusión}

Este proyecto implementa un sistema integral de análisis de sentimientos y verificación de
protocolos de atención mediante el uso de un Autómata Finito Determinista (AFD) y una tabla de
sentimientos. La arquitectura del sistema permite la identificación y clasificación de palabras
con carga emocional, así como la detección de palabras inapropiadas en las interacciones.
Además, el sistema verifica el cumplimiento de los elementos fundamentales del protocolo de
atención (saludo, identificación y despedida), proporcionando una evaluación detallada de las
respuestas del agente.

La flexibilidad del sistema para agregar o eliminar palabras de la tabla de sentimientos y
actualizar el AFD en tiempo real lo hace adaptable a nuevas necesidades o criterios específicos
de análisis. Esta capacidad de modificación y persistencia de datos asegura que el sistema se
pueda ajustar continuamente para mejorar la precisión de las evaluaciones y adaptarse a
diferentes contextos de uso.

En conjunto, este sistema representa una herramienta robusta y adaptable para el análisis
automatizado de interacciones en centros de contacto, con potencial de ser extendido o
integrado en aplicaciones más amplias de procesamiento de lenguaje natural (PLN) o de mejora de
calidad en la atención al cliente.


% Seccion de anexos
\section{Anexos}
El codigo fuente completo de este proyecto puede ser encontrado en:
\href{https://github.com/elias-gill/tp-compiladores}{https://github.com/elias-gill/tp-compiladores}.
\subsection{Codigo fuente}

\subsubsection{Tokenizador AFD}
\subsubsection{Tokenizador HashMap}
\subsubsection{Analizador de sentimientos}


\end{document}
